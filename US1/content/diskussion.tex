\section{Diskussion}
\label{sec:Diskussion}
Die Messung der Schallgeschwindigkeit mittels Impuls-Echo-Verfahren liefert einen Wert von
\begin{align*}
  c_{\text{acryl,e}} &= \SI{2745+-15}{\metre\per\second}
.
\end{align*}
Daraus folgt eine Abweichung zum Literaturwert \cite{acryl},
\begin{align*}
  c_{\text{acryl,lit}} &= \SI{2730}{\metre\per\second}
,
\end{align*}
von
\begin{align*}
  \increment c_{\text{acryl}} &= \SI{0.57}{\percent}
.
\end{align*}
Bei der Durchschallungs-Methode ergibt sich für die Phasengeschwindigkeit in Acryl
\begin{align*}
  c_{\text{acryl,d}} &= \SI{2684+-28}{\metre\per\second}
,
\end{align*}
die relative Abweichung zum Literaturwert beträgt hier
\begin{align*}
  \increment c_{\text{acryl,d}} &= \SI{1.68}{\percent}
.
\end{align*}
Die dritte Bestimmung der Schallgeschwindigkeit erfolgte über die Steigung der Ausgleichsrechnung, hier ergab sich ein Wert von
\begin{align*}
  c_{\text{acryl, a}} &= \SI{2760+-38}{\metre\per\second}

\end{align*}
mit einer relativen Abweichung von
\begin{align*}
  \increment c_{\text{acryl,a}} &= \SI{1.1+-1.4}{\percent}
.
\end{align*}
Die Impuse-Echo-Methode erweist sich dementsprechend als exakteste.\\
Die Anpassungsschicht wurde mittels y-Achsenabschnitt der Ausgleichsrechnung zu
\begin{align*}
  d &= \SI{-0.0005+-0.0013}{\metre}
.
\end{align*}
Dieser Wert scheint eine gute Beschreibung der Anpassungsschicht zu sein, da laut Literatur eine Dicke von durchschnittlich $\SI{0.5}{\milli\metre}$ bis zu $\SI{2.5}{\milli\metre}$ zu erwarten ist. \cite{anpassungsschicht}\\
Bei der Ausmessung der Plattendicken mittels Cepstrum und FFT konnte jeweils nur eine Länge
\begin{align*}
  s_\text{fft} &= \SI{0.0325+-0.0019}{\metre}
, \\
  s_\text{cep} &= \SI{0.032}{\metre}
,
\end{align*}
unbekannter Bedeutung bestimmt werden.
Vermutlich handelt es sich hierbei um die kombinierte Dicke beider Platten.\\
Die Untersuchung des Augenmodelles führte auf die Längen
\begin{align*}
  s_{1} &= \SI{0.022}{\metre}
, \\
  s_{2} &= \SI{0.017}{\metre}
, \\
  s_{3} &= \SI{0.070}{\metre}
, \\
\end{align*}
welche ungefähr die Größe skalierte Größe eines menschlichen Auges beschreiben.
