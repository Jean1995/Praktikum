\section{Aufbau und Durchführung}
\subsection{Aufbau}
\label{sec:Aufbau}
Zur Untersuchung der Materialien mittels Ultraschall wird im vorliegenden Versuch ein Ultraschallechoskop verwendet.
An diesem können bis zu zwei Ultraschallsonden unterschiedlicher Frequenzen angeschlossen werden, je nachdem ob das Impuls-Echo-Verfahren oder das Durchschallungsverfahren durchgeführt wird.
Der jeweilige Modus kann über einen Schalter an dem Echoskop gewählt werden.
Zudem sind weitere Funktionen zur Verstärkung (TGC) und Anpassung der Messung vorhanden.
Die gemessenen Daten werden vom Gerät an einen angeschlossenen Computer weitergeleitet, der diese auswertet.
Das vorhandene Computerprogramm ist in der Lage, die Signalstärke in Volt in Abhängigkeit von der Zeit graphisch darzustellen (A-Scan).
Bei der Eingabe der Schallgeschwindigkeit des untersuchten Mediums kann das Signal zudem in Abhängigkeit der Eindringtiefe dargestellt werden.
Mittels Cursor können somit die Signalamplituden und Laufzeiten bzw. Eindringtiefen sowie deren Differenzen möglichst genau bestimmt werden.
Zudem existiert ein Modus, der eine schnelle Fourier-Transformation durchführt, so dass das Frequenzspektrum und das Cepstrum der Messdaten dargestellt werden kann.
