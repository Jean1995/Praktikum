\section{Theorie}
\label{sec:Theorie}
Betrachtet man die Rotationsbewegung von ausgedehnten Körpern, so geht der Begriff der Punktmasse in das Trägheitsmoment $I$ über.
Der Begriff der Kraft wird durch das Drehmoment $M$ abgelöst, welches durch die Formel
\begin{equation}
  \vec{M} = \vec{F} \times \vec{r}
\end{equation}
beschrieben wird, wobei $\vec{r}$ den Abstandsvektor und $\vec{F}$ die angreifende Kraft beschreibt.
Das Trägheitsmoment berechnet sich bei der Rotation eines Systems von $n$ Massepunkten um eine feste Drehachse zu
\begin{equation}
  I = \sum_{i=1}^n r_{\perp, i}^2 m_i
  \label{eqn:tm1}
\end{equation}
beziehungsweise im Falle einer kontinuierlichen Masseverteilung zu
\begin{equation}
  I = \int_V^{} \vec{r_{\perp}}^2 \rho{\vec{r}} \symup{d}V,
  \label{eqn:tm2}
\end{equation}
wobei $r_{\perp, i}$ den Abstand der Masse $m_i$ zu der Drehachse ist.
Im Falle einer Rotation um eine beliebige Achse geht das Trägheitsmoment in den Trägheitstensor $I_{\alpha \beta}$ über.\\
Ist das Trägheitsmoment $I_s$ für die Drehung um die Schwerpunktsachse eines Körpers bekannt, so folgt mit dem Steinerschen Satz für das Trägheitsmoment $I$ für eine um $a$ parallel verschobene Achse
\begin{equation}
  I = I_s + m a^2,
  \label{eqn:steiner}
\end{equation}
wobei $m$ die Masse des Körpers beschreibt.\\
Im vorliegenden Versuch wird eine Feder betrachtet, welche bei einer Drehung aus der Ruhelage ein rücktreibendes Drehmoment von
\begin{equation}
  \lvert M \rvert = D \varphi
\end{equation}
bewirkt, sich also nach dem Hookschen Gesetz verhält.
Wird nun ein Körper mit Trägheitsmoment $I$ in diesem System ausgelenkt, ergibt sich die Differentialgleichung
\begin{equation}
  I \ddot{\varphi} = D \varphi,
\end{equation}
dessen Lösung dementsprechend einem harmonischen Oszillator mit Kreisfrequenz
\begin{equation}
  \omega = \sqrt{\frac{D}{I}}
\end{equation}
beziehungsweise Umlaufzeit von
\begin{equation}
  T = 2 \pi \sqrt{\frac{I}{D}} \label{eqn:zeiten}
\end{equation}
entspricht.

%\cite{sample}
