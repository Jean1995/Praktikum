\section{Diskussion}
\label{sec:Diskussion}
Die verwendete Methode, um in der Praxis Trägheitsmomente von Körpern zu messen, bestätigt sich im Hinblick auf den Zylinder.
Dort zeigt sich nur eine Abweichung von $\SI{1.5}{\percent}$.
Zudem decken sich die Intervalle in denen Praxis- und Theoriewert liegen.
Die Ergebnisse zur Kugel sehen jedoch etwas anders aus.\\
Es zeichnet sich eine Abweichung von $\SI{7.22}{\percent}$ zum Theoriewert ab.
Außerdem decken sich die Wertintervalle nicht.
Die Untersuchung der Puppe lässt mehrere Diskussionspunkte aufkommen.\\
Die Werte belegen große prozentuale Abweichungen von $\SI{35.17}{\percent}$ bei ausgestreckten Armen und $\SI{62.74}{\percent}$ bei anliegenden.
Bei ausgestreckten Armen fängt die Puppe bei der Rotation leicht zu schaukeln an, was sich negativ auf die Werte ausdrückt.
Bei gewünscht anliegenden Armen werden jene jedoch durch die bei der Rotation aufkommenden Zentripetalkraft nach außen gedrückt, wie bei beiden Posen auch die Beine.
Dies führt zu fehlerhaften Umlaufzeiten und somit zu ungenauen Trägheitsmomenten.
Das Messen der Umlaufzeiten liegt generell mittels Stoppuhr der menschlichen Reaktionszeit unterlegen.
Weiterhin sorgt die Näherung der Puppe durch einen Zylinderaufbau für die zu sehende hohe Ungenauigkeit von bis zu mehr als einem Viertel des Mittelwertes $(I_{\text{Theorie, Pose1}} = \input{build/traegheit_mensch_pose_1_theorie.tex})$ und noch nicht einmal mit dieser hohen Unschärfe wird der beobachtete Wert getroffen.
Bei sowohl den praktischen als auch den theoretischen Werten zeigt sich aber ein größeres Trägheitsmoment bei ausgestreckten Armen.
Dies bestätigt zumindest die Annahme, dass eine achsennahe Massenverteilung in einem niedrigererem Trägheitsmoment resultiert.\\
Insgesamt zeigen die Ergebnisse, dass die Versuchsanordnung nur für simple schwere Körper die Theorie unterstützt.
