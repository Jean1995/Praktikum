\section{Diskussion}
\label{sec:Diskussion}
Die verwendete Methode, um in der Praxis Trägheitsmomente von Körpern zu messen, bestätigt sich im Hinblick auf den Zylinder.
Dort zeigt sich nur eine Abweichung von $\SI{1.5}{\percent}$.
Zudem decken sich die Intervalle, in denen Experimental- und Theoriewert liegen.
Im Gegensatz dazu weichen die Ergebnisse für die Kugel stärker ab:\\
Es zeichnet sich eine Abweichung von $\SI{7.22}{\percent}$ zum Theoriewert ab.
Zudem decken sich sich die Fehlerintervalle hier nicht.\\
Die Untersuchung der Puppe lässt weitere Diskussionspunkte aufkommen:
Die Werte weisen große prozentuale Abweichungen von $\SI{35.17}{\percent}$ bei ausgestreckten Armen und $\SI{62.74}{\percent}$ bei anliegenden auf.
Bei ausgestreckten Armen fängt die Puppe bei der Rotation leicht zu schaukeln an, was sich negativ auf die Genauigkeit der Werte auswirkt.
Bei gewünscht anliegenden Armen werden jene jedoch durch die bei der Rotation aufkommenden Zentripetalkraft nach außen gedrückt, wie bei beiden Posen auch die Beine.
Dies führt zu fehlerhaften Umlaufzeiten und somit zu ungenauen Trägheitsmomenten.
Das Messen der Umlaufzeiten mittels Stoppuhr unterliegt generell der endlichen menschlichen Reaktionszeit.
Weiterhin führt die Näherung der Puppe durch einen Zylinderaufbau zu einer hohen Ungenauigkeit von bis zu einem Viertel des Mittelwertes. Trotz der entstehenden großen Unschärfe liegt der Theoriewert nicht im Fehlerintervall des Messwertes. \\
Weiterhin ist zu Beobachten, dass sich sowohl bei den experimentell als auch bei den theoretischen Werten ein größeres Trägheitsmoment bei ausgestreckten Armen ergibt.
Dies bestätigt zumindest die Annahme, dass eine achsennahe Massenverteilung in einem niedrigerem Trägheitsmoment resultiert.\\
Theoretisch hätte von den experimentell bestimmten Trägheitsmomenten das Eigenträgheitsmoment der Drillachse abgezogen werden müssen.
Dieses übertrifft jedoch die berechneten Trägheitsmomente jeglicher betrachteter Körper, so dass auf diese Korrektur verzichtet wurde.
Aufgrund der Tatsache, dass die Theoriewerte trotzdem eine hohe Ähnlichkeit zu den experimentell bestimmten Werten aufweisen, ist davon auszugehen, dass die Bestimmung des Eigenträgheitsmomentes fehlerhaft ist:
Dieses wurde vermutlich um einen Faktor der Größenordnung 1000 zu groß bestimmt.
Die Ursache dieses Fehlers konnte vom Autorenteam nicht ermittelt werden.\\
Insgesamt zeigen die Ergebnisse, dass die Versuchsanordnung nur für simple schwere Körper die Theorie unterstützt.
