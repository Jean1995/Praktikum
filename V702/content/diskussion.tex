\section{Diskussion}
\label{sec:Diskussion}
Im vorliegenden Versuch wurden 3 Halbwertszeiten bestimmt.
Die ermittelten Werte betrugen dabei
\begin{align*}
  T_\text{In} &= \input{build/halbzeit_indium.tex},\\
  T_\text{Rh,1} &= \input{build/halbzeit_rhodium.tex},\\
  T_\text{Rh,2} &= \input{build/halbzeit_rhodium2.tex}.
\end{align*}
Aus der Literatur \cite{halbzeiten} folgen die wahren Halbwertszeiten zu
\begin{align*}
  T_\text{In,lit} &= \input{build/halbzeit_indium_lit.tex},\\
  T_\text{Rh,1,lit} &= \input{build/halbzeit_rhodium_lit.tex},\\
  T_\text{Rh,2,lit} &= \input{build/halbzeit_rhodium2_lit.tex}.
\end{align*}
Dementsprechend betragen die Abweichungen der Messwerte zu den Literaturwerten jeweils
\begin{align*}
  \increment T_\text{In} &= \input{build/halbzeit_indium_rel.tex},\\
  \increment T_\text{Rh,1} &= \input{build/halbzeit_rhodium_rel.tex},\\
  \increment T_\text{Rh,2} &= \input{build/halbzeit_rhodium2_rel.tex}.
\end{align*}
Für Indium ergibt sich eine relative geringe Abweichung zu den Literaturwerten.
Zudem liegt der Literaturwert im Fehlerbereich des Messwertes.
Dies ist darauf zurückzuführen, dass der Zerfall von Indium im Gegensatz zum Zerfall des Rhodium nur durch eine Halbwertszeit charakterisiert ist und es somit keine Überlappungen gibt.
Zudem waren die gemessenen Impulse für gesamten Messzeitraum ausreichend groß, so dass aussagekräftige Werte gewonnen werden konnten.
Die Abweichungen vom Literaturwert sind mit der statistischen Natur der Zerfälle sowie gegebenenfalls einer nicht exakten Nullmessung, die ebenfalls statistischen Schwangungen unterliegt, zu erklären.\\
Bei der Messung des längerlebigen Isotops ergeben sich leicht größere Abweichungen.
Diese lassen sich neben dem bereits benannten statistischen Fehler darauf zurückführen, dass weiterhin ein Einfluss durch den Zerfall des anderen Nuklids bestand.
Außerdem war eine klare Trennung zwischen den beiden Messbereichen für die Fits nicht möglich.
Zuletzt waren die gemessenen Impulse für das längerlebige Isotops relativ gering, was die Messung anfälliger für statistische Fehler beispielsweise durch den Nulleffekt macht.
Trotzdem liegt der Literaturwert im Fehlerbereich des Messwertes.\\
Im Endeffekt ergibt sich der größte Fehler für die Bestimmung des kurzlebigen Rhodium Nuklids.
Dies ist darauf zurückzuführen, dass die Korrektur der Messwerte bereits dem oben diskutierten Messfehlern unterliegt.
Zudem treten alle bereits genannten Fehlerquellen hier ebenfalls auf.
