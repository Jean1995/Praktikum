\subsection{Durchführung}
\label{sec:durchführung}

\subsubsection{Vorbereitung}
Vor den Messungen muss zunächst der Nulleffekt bestimmt werden.
Ungewünschterweise wird durch das Zählrohr auch Hintergrundstahlung gemessen, welche bei den Messergebnissen weggeicht werden muss.
Um jene Zählrate zu bestimmen, wird ein Messdurchlauf über $\SI{900}{\second}$ ohne Probe durchgeführt.

\subsubsection{Messung der Proben}
Um auf die Halbwertszeiten zu kommen werden die Zerfälle in einem gewissen Zeitintervall über einen Zeitraum gemessen.
Dies wird für Indium in $\SI{220}{\second}$ Intervallen über eine Zeit von einer Stunde und für Rhodium in $\SI{17}{\second}$ Intervallen über eine Zeit von $\SI{12}{\minute}$ durchgeführt.
