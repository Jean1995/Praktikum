\section{Theorie}
\label{sec:Theorie}
Aus dem zweiten Hauptsatz der Thermodynamik geht hervor, dass es unmöglich ist, ohne äußere Einwirkung einem kälteren Reservoir Teile dessen Wärmemenge zu entziehen, um sie einem wärmeren Reservoir zuzuführen.
Der erste Hauptsatz beinhaltet jedoch die Möglichkeit, durch Zuführen zusätzlicher Arbeit auch den umgekehrten Prozess zu realisieren.
Die zusätzliche Wärmemenge $ Q_1 $ im wärmeren Reservoir $ R_1 $ entspricht nun der aufgewendeten Arbeit $ A $ addiert mit der aus dem kälteren Reservoir $ R_2 $ entzogenen Wärmemenge $ Q_2 $.
Eine Wärmepumpe kann dies leisten.\\
Die Effizienz wird durch ihre Güteziffer bestimmt.
Sie stellt sich zusammen aus der zugeführten Wärmemenge und der zu diesem Zweck aufgewendeten Arbeit:
\begin{equation}
  v = \frac{Q_1}{A}=\frac{T_1}{T_1-T_2}.
\end{equation}
Dabei ist $ T_1 $ die Temperatur in $R_1$ und $ T_2 $ die Temperatur in $ R_2 $.
Erfahrungsgemäß ist es jedoch schwierig ein absolut geschlossenes System innerhalb der Wärmepumpe zu gewährleisten, wodurch
\begin{equation}
  Q_1 < Q_2 + A
\end{equation}
ist.
Für die reale Güteziffer gilt dann
\begin{equation}
  v_{real} < \frac{T_1}{T_1-T_2}.
\end{equation}
\cite{sample}
