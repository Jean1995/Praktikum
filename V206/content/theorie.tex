\section{Zielsetzung}
Im vorliegenden Versuch wird eine Wärmepumpe untersucht.
Dazu sollen charakteristische Werte wie die Güteziffer, der Massendurchsatz sowie die Kompressorleistung bestimmt und mit den theoretischen Werten einer idealen Wärmepumpe verglichen werden.

\section{Theorie}
\label{sec:Theorie}
Aus dem zweiten Hauptsatz der Thermodynamik geht hervor, dass es unmöglich ist, ohne äußere Einwirkung einem kälteren Reservoir Teile dessen Wärmemenge zu entziehen, um sie einem wärmeren Reservoir zuzuführen.
Der erste Hauptsatz beinhaltet jedoch die Möglichkeit, durch Zuführen zusätzlicher Arbeit auch den umgekehrten Prozess zu realisieren. Laut ihm gilt
\begin{equation}
  Q_1 = A + Q_2
\end{equation}
wobei $Q_1$ die zusätzliche Wärmemenge im wärmeren Reservoir $ R_1 $, $A$ die aufgewendete Arbeit und $Q_2$ die dem kälteren Reservoir $R_2$ entzogenen Wärmemenge ist.
Eine Wärmepumpe kann dies leisten.\\
Die Effizienz der Wärmepumpe wird durch ihre Güteziffer bestimmt.
Sie stellt sich zusammen aus der zugeführten Wärmemenge und der zu diesem Zweck aufgewendeten Arbeit:
\begin{equation}
  v = \frac{Q_1}{A}=\frac{T_1}{T_1-T_2}.
\end{equation}
Dabei ist $ T_1 $ die Temperatur in $R_1$ und $ T_2 $ die Temperatur in $ R_2 $.
Damit diese Gleichung erfüllt werden kann, muss der Wärmetransport in einem reversiblen Prozess stattfinden.
Dies bedeutet, dass der durchgeführte Prozess jederzeit ohne Energieverluste umgekehrt ablaufen könnte.
Erfahrungsgemäß ist es jedoch schwierig ein absolut geschlossenes System innerhalb der Wärmepumpe zu gewährleisten, wodurch
\begin{equation}
  Q_1 < Q_2 + A
\end{equation}
ist.
Für die reale Güteziffer gilt dann
\begin{equation}
  v_{real} < \frac{T_1}{T_1-T_2}.
\end{equation}
\cite{sample}
