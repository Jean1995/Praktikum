\section{Diskussion}
\label{sec:Diskussion}
Auffällig bei der Auswertung ist vor allem der deutliche Unterschied zwischen $ v_{ideal} $ und $ v_{real} $, wie man der Tabelle \ref{tab:tabelle2} entnehmen kann.
Die Abweichung von ca. $82\%$ lässt sich auf verschiedene Faktoren zurückführen, dessen Ursprünge im Versuchsaufbau liegen.\\
Zunächst war die Isolierung der Reservoire an sich unzureichend und etwas marode.
Wir konnten die Wärme bzw. die Kälte an den Behältern deutlich erfühlen.
Dieses Isolierungsproblem führte zu einer äußeren Erwärmung von $R_2$, sowie einer äußeren Abkühlung von $R_1$, somit zu Energieverlust.
Dasselbe Phänomen konnten wir auch an den Kupferrohren wahrnehmen.
Außerdem ließ sich der Deckel der Behälter auf Grund der nicht ausreichend befestigten Halterungsplattformen nicht richtig anpassen.
Dies führte zu zusätzlichem Wärmeaustausch mit der Umgebung.\\
Ein anderes Problem liegt in der nicht richtig umgesetzten Kompressorleistung ($\SI{124,77}{\watt}$ im Vergleich zu den berechneten Werten in Tabelle \ref{tab:tabelle4}).
Zusätzlich komprimiert der Kompressor das Medium nur annähernd adiabatisch.\\
Abschließend wollen wir noch erwähnen, dass die ideale Güteziffer von einem reversiblen Carnot-Prozess ausgeht, welcher in dieser Form nicht realisiert werden kann.
Trotz dieser Rahmenbedingungen konnten wir durch das Experiment gute und aussagekräftige Werte gewinnen.
