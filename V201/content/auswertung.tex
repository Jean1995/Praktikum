\section{Auswertung}
\label{sec:Auswertung}

Für die Untersuchung des Dulong-Petitschen Gesetzes wird ein Mischungskalorimeter verwendet.
Die Temperaturmessung wird über ein Thermoelement durchgeführt.
Zuerst muss jedoch der Offset $U_{\text{offset}}$ des Elementes bestimmt werden.
Er beträgt $U_{\text{offset}}= \SI{-0,02}{\milli\volt}$.
Um die gemessenen Spannungen in Temperaturen umzurechnen, wird die Formel
\begin{equation}
  T = 25,157 \cdot U_{th} - 0,19 \cdot U_{th}²,
\end{equation}
$T$ in \SI{}{\celsius} und $U_{th}$ in \SI{}{\milli\volt}, verwendet.
Damit ergeben sich die Daten aus Tabelle \ref{tab:1}. \\
\\
\\
\begin{table}
  \centering
  \caption{Messdaten}
  \label{tab:1}
  \sisetup{table-format=1.2}
  \begin{tabular}{c c c c c c c c c}
    \toprule
    {Probe} & {$m_w [\si{\gram}]$} & {$m_k [\si{\gram}]$} & {$U_w [\si{\milli\volt}]$} & {$T_w [\si{\kelvin}]$} & {$U_k [\si{\milli\volt}]$} & {$T_k [\si{\kelvin}]$} & {$U_m [\si{\milli\volt}]$} & {$T_m [\si{\kelvin}]$} \\
    \midrule
    1.0000  & 11.00 \\
2.0000  & 12.00 \\
3.0000  & 13.00 \\
4.0000  & 14.00 \\
5.0000  & 15.00 \\
6.0000  & 16.00 \\
7.0000  & 17.00 \\
8.0000  & 18.00 \\
9.0000  & 19.00 \\
10.0000 & 20.00 \\

    \bottomrule
  \end{tabular}
\end{table}
\\
\\
Die Wärmekapazität des Dewargefäßes beträgt nun mit Hilfe von Zusammenhang \ref{eqn:7}
\begin{equation}
  c_gm_g = \frac{c_wm_2(T_2-T_m')-c_wm_1(T_m'-T_1)}{T_m'-T_1},
  \label{eqn:8}
\end{equation}
wie in der Durchführung beschrieben (\ref{sec:Durchfuehrung}).
Mit den folgenden Werten,
\begin{align}
  T_1 &= \SI{20,75}{\celsius} \\
  T_2 &= \SI{72,80}{\celsius} \\
  T_m'&= \SI{43,93}{\celsius} \\
  m_1 &= \SI{237,99}{\gram} \\
  m_2 &= \SI{234,33}{\gram} \\
  c_w &= \SI{4,18}{\joule\per\gram\per\kelvin},
\end{align}
ergibt sich nach \ref{eqn:8} für die Wärmekapazität des Dewargefäßes
\begin{equation}
  c_gm_g = \SI{224,84}{\joule\per\kelvin}.
\end{equation}

Nun können die spezifischen Wärmekapazitäten der Probekörper über
\begin{equation}
  c_k = \frac{(c_wm_w+c_gm_g)(T_m-T_w)}{m_k(T_k-T_m)}
\end{equation}
berechnet werden.
Aus den Messungen folgt auf die zweite Nachkommastelle gerundet
\begin{align}
  c_{\text{Blei}} &= \input{build/ck_blei.tex}\: \si{\joule\per\gram\per\kelvin} \label{Bleiwert:c}
\end{align}
für Blei, sowie
\begin{align}
  c_{\text{Graphit}} &= \input{build/ck_graphit.tex}\: \si{\joule\per\gram\per\kelvin} \label{Bullseye:c}
\end{align}
für Graphit.
Für Zinn ergeben die drei Messungen
\begin{align}
  c_{\text{Zinn,1}} &= \input{build/ck_zinn1.tex}\: \si{\joule\per\gram\per\kelvin} \label{Kackwert:c},\\
  c_{\text{Zinn,2}} &= \input{build/ck_zinn2.tex}\: \si{\joule\per\gram\per\kelvin},\\
  c_{\text{Zinn,3}} &= \input{build/ck_zinn3.tex}\: \si{\joule\per\gram\per\kelvin}.
\end{align}
Der Mittelwert der Zinnmessungen wird über die Formel
\begin{equation}
  \overline{x} = \frac{1}{N} \sum_{i=1}^N x_i \label{eqn:9}
\end{equation}
zu
\begin{align}
  \overline{c}_{\text{Zinn}} &= \input{build/ck_zinnm.tex}\: \si{\joule\per\gram\per\kelvin}.
\end{align}
bestimmt.
Somit folgt mit der Formel für die Standardabweichung
\begin{equation}
  \sigma = \sqrt{\overline{x²}-\overline{x}²} \label{eqn:10},
\end{equation}
dass der Fehler
\begin{align}
  c_{\sigma} &= \input{build/ck_zinna.tex}\: \si{\joule\per\gram\per\kelvin}
\end{align}
beträgt.\\
Nach Anwenden der Formeln \ref{eqn:5} und \ref{eqn:6},
\begin{equation}
  C_k = c_k M - 9\alpha² \kappa \frac{M}{\rho} T_m,
\end{equation}
folgen die Molwärmen bei konstantem Volumen zu
\begin{align}
  C_{\text{Graphit}} &= \input{build/C_graphit.tex}\: \si{\joule\per\mol\per\kelvin} \label{Bullseye:C},\\
  C_{\text{Blei}}    &= \input{build/C_blei.tex}\: \si{\joule\per\mol\per\kelvin} \label{Bleiwert:C},\\
  C_{\text{Zinn,1}}  &= \input{build/C_zinn1.tex}\: \si{\joule\per\mol\per\kelvin} \label{Kackwert:C},\\
  C_{\text{Zinn,2}}  &= \input{build/C_zinn2.tex}\: \si{\joule\per\mol\per\kelvin},\\
  C_{\text{Zinn,3}}  &= \input{build/C_zinn3.tex}\: \si{\joule\per\mol\per\kelvin}.
\end{align}
Die spezifischen Werte der Molmassen $M$, Dichten $\rho$, Kompresionsmodule $\kappa$ und Ausdehnungskoeffizienten $\alpha$ der einzelnen Stoffe sind im Anhang beigefügt.
Zuletzt ergibt der gemittelte Wert \ref{eqn:9} mit Abweichung \ref{eqn:10} der Zinnrechnung
\begin{align}
  C_{\text{Zinn}}  &= (\num{48,31+-29,93})\: \si{\joule\per\mol\per\kelvin}.
\end{align}
