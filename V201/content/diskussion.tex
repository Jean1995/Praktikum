\section{Diskussion}
\label{sec:Diskussion}
allgemeine Gaskonstante R = \SI{8,314}{\joule\per\mol\per\kelvin} \cite{codata}, somit sind 3R = \SI{24,942}{\joule\per\mol\per\kelvin}
%Bleiwert war erste Messung, die nicht sehr schnell und gut ging (Wackelkontaktkabel) in Bezug auf T_k, darum C_Blei = 44,12, Idealwert 26,4, Abweichung von 67%
%Zinnmessung, unser Wert 48,31 +- 29,93 trifft somit in Idealwert 27,11, erster Zinnwert war jedoch kaotisch, ohne ihn folgt (Messungen nach Austausch der Kabel mit Wackelkontakt)27,28 +- 4,08, was schon viel besser ist
%Graphitmessung war perfekt 8,51, Idealwert 8,53, macht eine Abweichung von 0,23%
% Insgesamt würde C = 3R bei Blei (82 units viel Masse, Gitterstruktur) und Zinn (50 units viel Masse, Gitterstruktur) recht gut zutreffen, bei Graphit jedoch nicht (6 units kleine Atommasse (1 unit mehr als Bor), daher wird 3R erst bei hohen Temperaturen erreicht, Kristallstruktur (kompliziertere Atomschwingung))
%Idealwerte von engl. Wikipedia heat capacity
