\section{Diskussion}
\label{sec:Diskussion}
%allgemeine Gaskonstante R = \SI{8,314}{\joule\per\mol\per\kelvin}, somit sind 3R = \SI{24,942}{\joule\per\mol\per\kelvin}
%Bleiwert war erste Messung, die nicht sehr schnell und gut ging (Wackelkontaktkabel) in Bezug auf T_k, darum C_Blei = 44,12, Idealwert 26,4, Abweichung von 67%
%Zinnmessung, unser Wert 48,31 +- 29,93 trifft somit in Idealwert 27,11, erster Zinnwert war jedoch kaotisch, ohne ihn folgt (Messungen nach Austausch der Kabel mit Wackelkontakt)27,28 +- 4,08, was schon viel besser ist
%Graphitmessung war perfekt 8,51, Idealwert 8,53, macht eine Abweichung von 0,23%
% Insgesamt würde C = 3R bei Blei (82 units viel Masse, Gitterstruktur) und Zinn (50 units viel Masse, Gitterstruktur) recht gut zutreffen, bei Graphit jedoch nicht (6 units kleine Atommasse (1 unit mehr als Bor), daher wird 3R erst bei hohen Temperaturen erreicht, Kristallstruktur (kompliziertere Atomschwingung))
%Idealwerte von engl. Wikipedia heat capacity
Nun werden die Auswertungsergebnisse in Bezug auf das Dulong-Petitsche Gesetz untersucht.
Vorher werden die Werte aber noch mit den Idealwerten aus der Literatur verglichen.
Zunächst zur Bleimessung.
Diese war unsere erste und sie lief in Bezug auf systematische Fehler nicht problemlos.
Da die Kabel unseres Thermoelements einen Wackelkontakt aufwiesen, fiel es uns nicht leicht, schnell und exakt die Temperatur des aufgeheizten Bleis zu messen.
Unser gemessener Wert für die Atomwärme von Blei beträgt deshalb $C_{\text{Blei}}=\SI{44,12}{\joule\per\mol\per\kelvin}$, eine Abweichung von $67\%$ vom Idealwert $C_{\text{Blei,ideal}}=\SI{26,4}{\joule\per\mol\per\kelvin}$.\\
Dieses Phänomen ereignete sich auch bei unserer ersten Zinnmessung \ref{Kackwert:C}, wodurch unser errechneter Wert auf $C_{\text{Zinn}}=(\num{48,31 +- 29,93})\: \si{\joule\per\mol\per\kelvin}$ fällt.
Nach der ersten Zinnmessung konnten wir auf nicht sporadische Kabel zurückgreifen.
Dies verbesserte unsere Messung, sodass wir ohne den Wert der ersten Messung auf $C_{\text{Zinn,gut}}=(\num{27,28 +- 4,08})\: \si{\joule\per\mol\per\kelvin}$ kommen.
Aber auch mit der ersten Messung liegt der ideale Wert $C_{\text{Zinn,ideal}}=\SI{26,4}{\joule\per\mol\per\kelvin}$
in der Fehlertoleranz.
Die Graphitmessung hingegen lief ohne Probleme, wodurch wir einen Wert von $C_{\text{Graphit}}=\SI{8,51}{\joule\per\mol\per\kelvin}$ präsentieren können.
Eine Abweichung von nur $0,23\%$ zum Idealwert von $C_{\text{Graphit,ideal}}=\SI{8,53}{\joule\per\mol\per\kelvin}$.\cite{wiki}\\
Es bleibt zu erwähnen, dass das Dulong-Petitsche Gesetz besonders bei schweren Atomen im Gittermodell präzise ist.
Die allgemeine Gaskonstante beträgt $R = \SI{8,314}{\joule\per\mol\per\kelvin}$ \cite{codata}, somit sind $3R = \SI{24,942}{\joule\per\mol\per\kelvin}$.
Im Bezug auf Blei sollte es also zustimmen, da es mit 82 units ein relativ schweres Element ist und eine Gitterstruktur aufweist.
Verglichen mit dem Idealwert lässt sich das Gesetz bestätigen, wir können dies aber mithilfe unserer Messung nicht ausreichend untermalen.
Ganz im Gegensatz zu unserer Zinnmessung ohne den ersten Wert.
Zinn ist auch ein schweres Element (50 units) mit solider Gitterstruktur und unser Wert bestätigt das Dulong-Petitsche Gesetz.
Graphit hingegen gehört zu den Halbmetallen, es ist mit 6 units (1 unit mehr als Bor) sehr leicht und es hat eher eine kristalline Struktur.
Es unterliegt laut unserer Messung nicht dem Gesetz.\\
Abschließend lässt sich sagen, dass sich das Dulong-Petitsche Gesetz quasi als eine Faustregel für schwere Metalle eignet.
Um aber genaue Werte zu errechnen sollte auf den quantenmechanischen Zusammenhang zurückgegriffen werden.
