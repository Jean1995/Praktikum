\section{Theorie}
\label{sec:Theorie}
Das Dulong-Petitsche Gesetz besagt, dass die Molwärme eines aus Atomgittern bestehenden Stoffes material- und temperaturunabhängig ist.
Sie beträgt immer
\begin{equation}
  C = 3R,
  \label{eqn:1}
\end{equation}
wobei $C$ die Molwärme/Atomwärme und $R$ die allgemeine Gaskonstante ist.
%Bei Versuchen ist es praktischer von der Wärmekapazität $C$ auszugehen, für die
%\begin{equation}
%  C = C_mn
%\end{equation}
%gilt.
%Dabei ist $n$ die Stoffmenge.
Zwischen der Atomwärme bei konstantem Volumen und der Wärmemenge $Q$ eines Stoffes besteht der Zusammenhang
\begin{equation}
  C_V = \left(\frac{\symup{d}Q}{\symup{d}T}\right)_V
\end{equation}
Da bei diesem isochoren Zusammenhang keine Arbeit verrichtet wird, ist laut dem 1. Hauptsatz der Thermodynamik
\begin{equation}
  \symup{d}U = \symup{d}Q + \symup{d}A
  \label{eqn:2}
\end{equation}
$\symup{d}Q$ identisch $\symup{d}U$ und es folgt
\begin{equation}
  C_V = \left(\frac{\symup{d}U}{\symup{d}T}\right)_V.
\end{equation}\\
Die innere Energie $U$ lässt sich jedoch über die klassische Physik oder durch die Quantenmechanik beschreiben.
Allgemein setzt sich jene Energie aus der kinetischen und potentiellen Energie der Atome zusammen, welche auf Grund von Gitterkräften im Material nur eine Schwingung der Frequenz $\omega$ ausführen können.
Die innere Energie wird nun gemittelt.\\
In der klassischen Physik, auf die sich das Dulong-Petitsche Gesetz stützt, wird davon ausgegangen, dass die Atome kontinuierlich Energie austauschen können.
Demnach folgt
\begin{equation}
  \langle U_{kl} \rangle = 3RT
  \label{eqn:3}
\end{equation}
für die gemittelte innere Energie.\\
Die Quantenmechanik andererseits verlangt, dass die Energien nur in diskreten Portionen, also gequantelt, ausgetauscht werden können.
Nach einem komplizierteren Zusammenhang zwischen $U$ und $T$ ergibt sich
\begin{equation}
  \langle U_{qu} \rangle = \frac{3N_L\hbar\omega}{exp(\hbar\omega/kT)-1}
  \label{eqn:4}
\end{equation}
mit $N_L = \SI{6,02e23}{\per\mol}$ (Loschmidtsche Zahl), der Plankschen Konstante $\hbar$ und der Boltzmannkonstante $k$.\\
Wird der quantenmechanische Ausdruck \ref{eqn:4} für $T\to\infty $ getaylort, ergibt sich ungefähr der klassische Ausdruck \ref{eqn:3}, da für hohe Temperaturen, also viel Teilchenbewegung, der Energieaustausch annähernd kontinuierlich wird.
Ansonsten gilt
\begin{equation}
  \langle U_{qu} \rangle < \langle U_{kl} \rangle = C_VT.
\end{equation}
Der klassische Begriff verfehlt die realen Werte bei geringen Temperaturen und geringen Atommassen, da er von beliebig kleinem Energieaustausch ausgeht.\\
Um nun den Wert der Atomwärme zu überprüfen wird am besten ein Experiment gewählt, das auf einer isobaren Bedingung basiert.
Daher eignet sich ein Kalorimeter.
Um später wieder von dem errechneten $C_P$ auf $C_V$ zu kommen wird der Zusammenhang
\begin{equation}
  C_P - C_V = 9 \alpha² \kappa V_0 T,
  \label{eqn:5}
\end{equation}
ausgenutzt.
Dabei ist $\alpha$ der lineare Ausdehnungskoeffizient, $\kappa$ das Kompressionsmodul und $V_0$ das Molvolumen.
Zudem geht man im Versuch von der spezifischen Wärmekapazität $c$ aus, für die
\begin{equation}
  cm = Cn
  \label{eqn:6}
\end{equation}
gilt.
Dabei ist $n$ die Stoffmenge und $m$ die Masse des gesamten zu untersuchenden Körpers.
Das Verfahren beim Mischungskalorimeter basiert wieder auf dem 1. Hauptsatz der Thermodynamik \ref{eqn:2}.
Wenn man nun zwei Materialien mit unterschiedlichen Temperaturen vermischt, gilt
\begin{equation}
  Q_1 = Q_2,
\end{equation}
da $\symup{d}A$ jeweils Null ist und keine Energie aus dem System verloren geht.
Daraus folgt weiterhin
\begin{equation}
  c_1m_1(T_1-T_m) = c_2m_2(T_m-T_2).
  \label{eqn:7}
\end{equation}
Somit ist es möglich, wenn die Masse und die spezifische Wärmekapazität von einem der Stoffe bekannt ist, mit Hilfe der Mischtemperatur $T_m$ auf die Wärmekapazität des jeweils anderen zu schließen.
Durch \ref{eqn:7}, \ref{eqn:5} und \ref{eqn:6} kann das Dulong-Petitsche Gesetz überprüft werden.


\cite{sample}
