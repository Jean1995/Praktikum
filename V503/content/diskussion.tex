\section{Diskussion}
\label{sec:Diskussion}

In diesem Protokoll werden drei signifikante Stellen angegeben, da es sich bei den zu Untersuchenden Größen um sehr bekannte und zudem recht große, beziehungsweise kleine Größen handelt.
Die Ergebnisse des Versuches sind in Tabelle \ref{tab:Ende} wiedergegeben, wobei die Literaturwerte \cite{chemie}
\begin{align*}
  e_{0\text{,lit}} &= 1,602 \cdot 10^{-19}\si{\coulomb},\\
  N_{\text{a,lit}} &= 6,022 \cdot 10^{23}\si{\per\mol}\\
\end{align*}
betragen.
\input{build/Tabelle_Ende_texformat.tex}
Die Abweichungen sind recht groß, jedoch tragbar im Anbetracht des Versuchs, bei dem nicht zuletzt zwischenmenschliche Reaktionszeit zu einer Unsicherheitsquelle gehört, begründet darin, dass eine Person die Teilchenbewegung steuert, während die andere nach Anweisung der ersten Person die Zeit stoppen muss.\\
Eine weitere Fehlerquelle ist, dass der Apparat nicht ausreichend windgeschützt ist.
So sorgt der kleinste Windstoß, ausgelöst beispielsweise durch passierende Mitpraktikanten, für eine Verfälschung der Messwerte.\\
Eine Verbesserung der Werte kann mit mehreren Messreihen gewährleistet werden.
Die Schwierigkeit des Versuchs liegt aber darin, ein Teilchen zu finden, welches alle Anforderungen \ref{sec:durchfuehrung} erfüllt und dieses für mehrere Messwerte im Auge zu behalten.\\
Alles in allem ist der Versuch, entgegen der Erwartungen der Experimentatoren, geglückt.
