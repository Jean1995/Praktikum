\subsection{Durchführung}
\label{sec:durchfuehrung}
Zuerst wird über die Libelle überprüft, ob der Versuch senkrecht zur Erdanziehungskraft steht.
Ist dies gegeben, wird die Millikan-Kammer mit Öltröpfchen bestäubt.
Danach wird ein Öltröpfchen auf dem Bildschirm angepeilt, welches zwei Kriterien erfüllt.
Es muss erstens sinken und zweitens negativ geladen sein.
Ob das Teilchen geladen ist, wird durch wiederholtes umpolen der Kondensatorplatten bestimmt.
Gibt es kein geladenes Teilchen, wird die Luft durch das Freigeben des Alpha-Präparates ionisiert.
Ist ein geeignetes Tröpfchen gefunden, wird einmalig dessen Sinkzeit über eine festgelegte Strecke gemessen.
Danach wird dessen Aufsteigezeit mit entgegen des Schwerefeldes gepolter Kondensatorplatten aufgenommen.
Das Feld wird umgepolt und es wird erneut die Sinkzeit gemessen.
Dies wird für ein Teilchen, solange es fokussiert ist, so oft wie möglich wiederholt.
Dies wird so häufig es geht für viele Teilchen unter verschiedenen Kondensatorspannungen durchgeführt.
Währenddessen wird die Temperatur notiert, unter die die Messreihen aufgenommen werden.
