\section{Auswertung}
\label{sec:Auswertung}

Zur Bestimmung der Ladung der Öltröpfchen werden Messungen für die Kondensatorspannungen
\begin{align*}
U_1 &= \SI{190}{\volt},\\
U_2 &= \SI{302}{\volt},\\
U_3 &= \SI{250}{\volt}
\end{align*}
durchgeführt.
Die Temperaturen der Messapparatur bestimmen sich aus den gemessenen Werten des Thermowiderstandes unter Verwendung der in Abbildung \ref{tab:thermo} angegebenen Tabelle zu
\begin{align*}
R_1 &= \SI{1.71}{\mega\ohm},\\
T_1 &= \SI{32}{\celsius},\\
R_2 &= \SI{1.67}{\mega\ohm},\\
T_2 &= \SI{33}{\celsius}.
\end{align*}
\begin{figure}
  \centering
  \includegraphics[height=5cm]{ressources/thermo.png}
  \caption{Thermistor-Widerstandstabelle \cite{skript}.}
  \label{tab:thermo}
\end{figure}
Bei der Betrachtung der Werte fällt auf, dass vereinzelt eine starke Abweichung der Messzeiten während einer einzigen Messreihe aufgetreten ist.
Diese Messwerte können dementsprechend nicht verwendet werden.
Zudem werden die Messwerte verworfen, wenn die Bedingung
\begin{align*}
  v_{\text{ab}} - v_{\text{auf}} = 2 v_0
\end{align*}
nicht erfüllt wird.
Somit werden die in den Tabellen \ref{table:1} bis \ref{table:6} angegebenen Messwerte verwendet.
\begin{table}
    \centering
    \caption{Bestimmung der Schallgeschwindigkeit mittels Durchschallungs-Methode.}
    \label{tab:1}
    \sisetup{parse-numbers=false}
    \begin{tabular}{
	S[table-format=2.2]
	S[table-format=2.1]
	S[table-format=4.2]
	}
	\toprule
	{$h_{\text{zylinder}} \:/\: 10^{-3} \si{\metre}$}		& {$\increment t \:/\: 10^{-6} \si{\second} $}		& 
	{$c_\text{Acryl} \:/\: \si{\metre\per\second} $}		\\ 
	\midrule
    31.30 & 11.5 & 2709.96 \\
61.50 & 22.8 & 2697.37 \\
80.55 & 30.4 & 2645.32 \\

    \bottomrule
    \end{tabular}
    \end{table}

\input{build/Tabelle_2_texformat.tex}
\input{build/Tabelle_3_texformat.tex}
\input{build/Tabelle_4_texformat.tex}
\input{build/Tabelle_5_texformat.tex}
\input{build/Tabelle_6_texformat.tex}
Von diesen Zeiten wird für jede Messreihe, also für jedes betrachtete Teilchen, der Mittelwert gebildet.
Die betrachtete Strecke beträgt jeweils $s = \SI{1}{\milli\metre}$, woraus sich mit den gemittelten Zeiten die Geschwindigkeiten $v_{\text{ab}}$ und $v_{\text{auf}}$ ergeben.


%\clearpage

%Von diesen Zeiten wird für jede Messreihe, also für jedes betrachtete Teilchen, der Mittelwert gebildet,
%Der betrachtete Strecke beträgt jeweils $s = \SI{1}{\milli\metre}$, woraus sich mit den gemittelten Zeiten die Geschwindigkeiten $v_{\text{ab}}$ und $v_{\text{auf}}$ ergeben.
\subsection{Einfache Bestimmung der Elementarladung}

Die daraus nach Formel \eqref{eqn:1} resultierenden Ladungen sind in Tabelle \ref{table:q} dargestellt.
Dabei wird die Beziehung
\begin{equation}
  E = \frac{U}{d},
\end{equation}
welche für das elektrische Feld $E$ eines Plattenkondensators gilt, ausgenutzt.
Die verwendeten Konstanten sind aus \cite{chemie}, \cite{skript} entnommen und betragen
\begin{align*}
  \rho_{\text{Oel}}  &= \SI{886}{\kilo\gram\per\metre\tothe{3}},\\
  \rho_{\text{Luft}} &= \SI{1,1644}{\kilo\gram\per\metre\tothe{3}},\\
  g                  &= \SI{9,81}{\metre\per\second},\\
  d                  &= \SI{7,6250 \pm 0,0051}{\milli\metre},\\
\end{align*}
wobei die Viskosität $n$ von Luft je nach Temperatur aus dem Diagramm \ref{luft_n_shit} entnommen wird.
\begin{figure}
  \centering
  \includegraphics[height=6cm]{ressources/luft.png}
  \caption{Viskosität von Luft \cite{skript}.}
  \label{luft_n_shit}
\end{figure}
Zudem wird der Fehler der Ladungen nach Formel \eqref{eq:gauss} angegeben.
\input{build/Tabelle_q_texformat.tex}

Zusätzlich sind die Ergebnisse der Ladungen in Abbildung \ref{plot:1} graphisch dargestellt.
\begin{figure}
  \centering
  \includegraphics[height=8cm]{build/ladungen.pdf}
  \caption{Ermittelte Ladungen aus Tabelle \ref{table:q}.}
  \label{plot:1}
\end{figure}

Wird nun der größte gemeinsame Teiler dieser Werte bestimmt, so ergibt sich ein ermittelter Wert für die Elementarladung von
\begin{align*}
  e_0 = \input{build/e_0.tex}.
\end{align*}
Die Berechnung erfolgt mithilfe eines im Python implementierten, dem Problem angepassten euklidischen Algorithmus (vgl. Abbildung \ref{kot}).

Die Avogadro-Konstante wird durch die Formel
\begin{equation}
  N_{\text{a}} = \frac{F}{e_0},
\end{equation}
mit der Faraday-Konstante $F = \SI{96485,3365}{\coulomb\per\mol}$ \cite{chemie}, beschrieben.
Dadurch wird die Avogadro-Konstante auf
\begin{align*}
  N_{a} = \input{build/N_a.tex}
\end{align*}
bestimmt.


\subsection{Bestimmung der Elementarladung mit Korrekturfaktor}

Um die Ladungen nach Formel \eqref{blubb} zu bestimmen, muss zunächst der Radius $r$ des Tröpfchens bekannt sein.
Jener wird über die Formel \eqref{blubbr}, mit den wie zuvor auch verwendeten Parametern, ermittelt.
Die Radien, und die nach \ref{eq:gauss} berechneten Fehler der in der Messreihe verwendeten Öltröpfchen, sind in Tabelle \ref{tabella} einzusehen.

\input{build/Tabelle_r_texformat.tex}

Mit den Parametern aus \cite{skript}, \cite{normaldruck}
\begin{align*}
  B &= 6,17\cdot 10^{-3}\text{Torr} \:\: \si{\centi\metre},\\
  p &= 1013,25\cdot10^2\si{\pascal}
\end{align*}
ergeben sich die somit neu bestimmten Ladungen in Tabelle \ref{table:q_neu}, welche auch in Abbildung \ref{plot:2} graphisch dargestellt sind.

\input{build/Tabelle_q_neu_texformat.tex}


\begin{figure}[H]
  \centering
  \includegraphics[height=8cm]{build/ladungen_neu.pdf}
  \caption{Korrigierte Ladungen.}
  \label{plot:2}
\end{figure}

Nach analogem Vorgehen ergibt sich eine Elementarladung von
\begin{align*}
  e_{0,\text{neu}} = \input{build/e_0_neu.tex}
\end{align*}
und die Avogadro-Konstante beträgt für die korrigierte Ladung
\begin{align*}
  N_{a,\text{neu}} = \input{build/N_a_neu.tex}.
\end{align*}



% % Examples
% \begin{equation}
%   U(t) = a \sin(b t + c) + d
% \end{equation}
%
% \begin{align}
%   a &= \input{build/a.tex} \\
%   b &= \input{build/b.tex} \\
%   c &= \input{build/c.tex} \\
%   d &= \input{build/d.tex} .
% \end{align}
% Die Messdaten und das Ergebnis des Fits sind in Abbildung~\ref{fig:plot} geplottet.
%
% %Tabelle mit Messdaten
% \begin{table}
%   \centering
%   \caption{Messdaten.}
%   \label{tab:data}
%   \sisetup{parse-numbers=false}
%   \begin{tabular}{
% % format 1.3 bedeutet eine Stelle vorm Komma, 3 danach
%     S[table-format=1.3]
%     S[table-format=-1.2]
%     @{${}\pm{}$}
%     S[table-format=1.2]
%     @{\hspace*{3em}\hspace*{\tabcolsep}}
%     S[table-format=1.3]
%     S[table-format=-1.2]
%     @{${}\pm{}$}
%     S[table-format=1.2]
%   }
%     \toprule
%     {$t \:/\: \si{\milli\second}$} & \multicolumn{2}{c}{$U \:/\: \si{\kilo\volt}$\hspace*{3em}} &
%     {$t \:/\: \si{\milli\second}$} & \multicolumn{2}{c}{$U \:/\: \si{\kilo\volt}$} \\
%     \midrule
%     \input{build/table.tex}
%     \bottomrule
%   \end{tabular}
% \end{table}
%
% % Standard Plot
% \begin{figure}
%   \centering
%   \includegraphics{build/plot.pdf}
%   \caption{Messdaten und Fitergebnis.}
%   \label{fig:plot}
% \end{figure}
%
% 2x2 Plot
% \begin{figure*}
%     \centering
%     \begin{subfigure}[b]{0.475\textwidth}
%         \centering
%         \includegraphics[width=\textwidth]{Abbildungen/Schaltung1.pdf}
%         \caption[]%
%         {{\small Schaltung 1.}}
%         \label{fig:Schaltung1}
%     \end{subfigure}
%     \hfill
%     \begin{subfigure}[b]{0.475\textwidth}
%         \centering
%         \includegraphics[width=\textwidth]{Abbildungen/Schaltung2.pdf}
%         \caption[]%
%         {{\small Schaltung 2.}}
%         \label{fig:Schaltung2}
%     \end{subfigure}
%     \vskip\baselineskip
%     \begin{subfigure}[b]{0.475\textwidth}
%         \centering
%         \includegraphics[width=\textwidth]{Abbildungen/Schaltung4.pdf}    % Zahlen vertauscht ... -.-
%         \caption[]%
%         {{\small Schaltung 3.}}
%         \label{fig:Schaltung3}
%     \end{subfigure}
%     \quad
%     \begin{subfigure}[b]{0.475\textwidth}
%         \centering
%         \includegraphics[width=\textwidth]{Abbildungen/Schaltung3.pdf}
%         \caption[]%
%         {{\small Schaltung 4.}}
%         \label{fig:Schaltung4}
%     \end{subfigure}
%     \caption[]
%     {Ersatzschaltbilder der verschiedenen Teilaufgaben.}
%     \label{fig:Schaltungen}
% \end{figure*}
