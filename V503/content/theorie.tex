\section{Zielsetzung}
Im Jahre 1910 ist es Robert Millikan gelungen, mithilfe des berühmten Millikan-Versuchs die Elementarladung $e_0$ möglichst genau zu bestimmen.
Dieses Experiment wird im folgenden durchgeführt.

\section{Theorie}
\label{sec:Theorie}
Zur Bestimmung der Elementarladung $e_0$, welche nur als diskrete Vielfache auftreten kann, werden geladene Öltröpfchen verwendet. Diese befinden sich zwischen den Platten eines Plattenkondensators.
\subsection{Kräftegleichgewicht im ausgeschalteten Zustand}
Im ausgeschalteten Zustand des Kondensators befindet sich das Öltröpfchen, sobald es sich mit einer konstanten Geschwindigkeit $v_0$ nach unten bewegt, im Kräftegleichgewicht
\begin{equation}
  \vec{F_\text{g}} + \vec{F_\text{R}} = 0.
\end{equation}
Hierbei wird die Gravitationskraft durch
\begin{equation}
  \vec{F_\text{g}} = \frac{4 \pi}{3} r^3 \rho_\text{L} g
\end{equation}
mit dem Tröpfchenradius $r$, der Erdbeschleunigung $g$ und der Dichte von Luft $\rho_\text{L}$ beschrieben.
Als Reibung wird eine Stokes-Reibung angenommen, welche durch
\begin{equation}
  \vec{F_\text{R}} = - 6 \pi r \eta_\text{L} v_0
\end{equation}
mit der Viskosität von Luft $\eta_\text{L}$ beschrieben wird.
Zu beachten ist, dass die Stokes-Reibung in dieser Form nur für den Fall gültig ist, dass das Tröpfchen einen größeren Durchmesser als die mittlere freie Weglänge $\lambda$ von Luft besitzt.
Ansonsten muss die Cunnigham-Korrektur verwendet werden, so dass für die korrigierte Viskosität
\begin{equation}
  \tilde{\eta_\text{L}} = \eta_\text{L} \Bigl( \frac{1}{1 + B \frac{1}{pr}}  \Bigr)
\end{equation}
mit dem Luftdruck $p$ sowie dem Korrekturterm $B$ gilt.
\subsection{Kräftegleichgewicht bei eingeschaltetem Kondensator}
Sobald der Kondensator eingeschaltet wird, existiert ein elektrisches Feld, welches je nach Polung eine Kraft
\begin{align*}
  \vec{F_\text{e}} = \pm qE
\end{align*}
auf das Öltröpfchen ausübt.
Falls beispielsweise die obere Platte positiv geladen ist, wirkt die Kraft des elektrischen Feldes nach oben, während die Reibungskraft nach unten wirkt.
Aus den sich ergebenden Kräftegleichgewichten kann die Ladung zu
\begin{equation}
  q = 3 \pi \eta_\text{L} \sqrt{ \frac{9}{4} \frac{\eta_\text{L}}{g} \frac{ (v_\text{ab} - v_\text{auf}  )  }{\rho_\text{oel} - \rho_\text{luft}} } \frac{v_\text{ab} + v_\text{auf}}{E} \label{eqn:1}
\end{equation}
bestimmt werden, wobei $E$ die Stärke des elektrischen Feldes, $\rho_\text{oel}$ die Dichte der verwendeten Öltröpfchen, $v_\text{ab}$ die gleichmäßige Geschwindigkeit im Kräftegleichgewicht nach unten  sowie $v_\text{auf}$ im Kräftegleichgewicht nach oben ist.
Das Radius berechnet sich zu
\begin{equation}
  r = \sqrt{ \frac{4}{9} \frac{ \eta_\text{L} ( v_\text{ab} - v_\text{auf} ) }{g (\rho_\text{oel} - \rho_\text{luft}) }   }. \label{blubbr}
\end{equation}
Mit der Korrektur beträgt die Ladung somit
\begin{equation}
  q = q_0 \bigl( 1 + \frac{B}{p r}  \bigr)^{-\frac{2}{3}}. \label{blubb}
\end{equation}
%\cite{sample}

% 2x2 Plot
% \begin{figure*}
%     \centering
%     \begin{subfigure}[b]{0.475\textwidth}
%         \centering
%         \includegraphics[width=\textwidth]{Abbildungen/Schaltung1.pdf}
%         \caption[]%
%         {{\small Schaltung 1.}}
%         \label{fig:Schaltung1}
%     \end{subfigure}
%     \hfill
%     \begin{subfigure}[b]{0.475\textwidth}
%         \centering
%         \includegraphics[width=\textwidth]{Abbildungen/Schaltung2.pdf}
%         \caption[]%
%         {{\small Schaltung 2.}}
%         \label{fig:Schaltung2}
%     \end{subfigure}
%     \vskip\baselineskip
%     \begin{subfigure}[b]{0.475\textwidth}
%         \centering
%         \includegraphics[width=\textwidth]{Abbildungen/Schaltung4.pdf}    % Zahlen vertauscht ... -.-
%         \caption[]%
%         {{\small Schaltung 3.}}
%         \label{fig:Schaltung3}
%     \end{subfigure}
%     \quad
%     \begin{subfigure}[b]{0.475\textwidth}
%         \centering
%         \includegraphics[width=\textwidth]{Abbildungen/Schaltung3.pdf}
%         \caption[]%
%         {{\small Schaltung 4.}}
%         \label{fig:Schaltung4}
%     \end{subfigure}
%     \caption[]
%     {Ersatzschaltbilder der verschiedenen Teilaufgaben.}
%     \label{fig:Schaltungen}
% \end{figure*}
