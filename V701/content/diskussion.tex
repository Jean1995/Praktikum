\section{Diskussion}
\label{sec:Diskussion}

Die Messung der mittleren Reichweite von Alpha-Strahlung über die Impulse ergibt
\begin{align*}
  x_{\text{mit}} = \input{build/x_mittel_1.tex}.
\end{align*}
Daraus resultiert eine Energie von
\begin{align*}
  E_{\text{mit}} = \input{build/E_mittel_1.tex}.
\end{align*}
Auffällig ist, dass die mittlere Reichweite relativ gering ist, da nach Literaturwerten Alpha-Strahlung eine Reichweite von etwa $\SI{4}{\centi\metre}$ bis $\SI{7}{\centi\metre}$ besitzt.
Die mittlere Reichweite sollte demnach im Bereich von $\SI{4}{\centi\metre}$ liegen.
Der Energieverlust von
\begin{align*}
  -\frac{\symup{d}E}{\symup{d}x} = \input{build/parameter_a_1.tex}.
\end{align*}
bestätigt die gemessenen Reichweiten, da kein Teilchen über $\SI{2}{\centi\metre}$ hinaus geht.\\
Bei der zweiten Messung kann wegen des geringen Abstandes keine mittlere Reichweite bestimmt werden.
Der Energieverlust beträgt hier aber
\begin{align*}
  -\frac{\symup{d}E}{\symup{d}x} = \input{build/parameter_a_2.tex}.
\end{align*}
Dies ist eine Abweichung von $\SI{35}{\percent}$ zur ersten Messung.\\
Die Zerfallsmessung übergibt eine Standardabweichung von
\begin{align*}
  \sigma = \input{build/sigma_stat.tex}.
\end{align*}
Die Messung scheint sich weder gut einer Gauß- noch einer Poissonverteilung anzupassen.
Da es sich hierbei um einen statistischen Prozess handeln sollte, kann es sein, dass über einen längeren Zeitraum hätte gemessen werden sollen.
