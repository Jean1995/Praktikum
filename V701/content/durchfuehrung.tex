\subsection{Durchführung}
\label{sec:durchführung}
\subsubsection{Einstellung der Diskriminatorschwelle}
Zunächst wird, um die Impulsschwelle richtig einzustellen und somit Hintergrundrauschen zu minimieren, die Diskriminatorschwelle eingestellt.
Dazu wird bei Atmosphärendruck die Strahlungsquelle weitestmöglich vom Detektor entfernt und die Schwelle am Vierkanalanalysator so eingestellt, dass keine Impulse mehr in einem hinreichend großen Zeitraum detektiert werden.
Daraufhin wird der Abstand solange verringert, bis wieder erste Impulse wahrgenommen werden.

\subsubsection{\texorpdfstring{Bestimmung der Reichweite von $\alpha$-Strahlung}{Bestimmung der Reichweite von Alpha-Strahlung}.}
Zunächst wird der Glaszylinder mithilfe der Vakuumpumpe evakuiert.
Für eine konstante Messzeit wird die mittlere Teilchenenergie anhand des Channels mit der höchsten Zählrate sowie die Gesamtzählrate bestimmt.
Diese Messung wird für verschiedene Drücke bis zum Atmosphärendruck wiederholt.
Zudem wird der gesamte Messvorgang für einen weiteren, geringeren Abstand wiederholt.

\subsubsection{Bestimmung der Statistik des radioaktiven Zerfalles}
Für einen festen Abstand der Strahlungsquelle zum Detektor wird bei evakuiertem Glaszylinder die Zählrate für ein  konstantes Zeitintervall bestimmt.
Dies wird für eine hinreichend große Anzahl an Messungen wiederholt, um eine aussagekräftige Statistik zu erhalten.
