\section{Theorie}
\label{sec:Theorie}

Protonen und Neutronen werden in Atomkernen durch die starke Kraft zusammengehalten.
Diese besitzt jedoch nur eine geringe Reichweite.
Aufgrund dessen überwiegt die abstoßende Coulombkraft zwischen den Protonen die anziehende starke Kraft, wodurch nun beispielsweise ein Helium-4-Atomkern ausgesendet wird.
Dies ist ein $\alpha$-Teilchen.\\
Ein solcher Zerfallsprozess kann auch über Tunneleffekte ablaufen.
Die Energie von $\alpha$-Teilchen wird über dessen Reichweite bestimmt bzw. abgeleitet.
Es verliert über zwei Aspekte seine Energie.\\
Zunächst über elastische Stöße mit dem Medium, welches es durchläuft.
Dieser Prozess, als Rutherford-Streuung bekannt, spielt für den Energieverlust eine nebensächliche Rolle, da die Größe der Atomkerndichte im Material sehr gering ist und somit ein Zusammenstoß relativ unwahrscheinlich ist.\\
Den zweiten, weitaus wichtigeren, Aspekt stellen Ionisationsprozesse, sowie Anregung und Dissoziation von Molekülen im Material dar.
Dabei hängt auch hier der Energieverlust pro Wegstück von der Dichte des Materials und der Energie $E_{\alpha}$ der $\alpha$-Strahlung ab.
Bei einer hohen Geschwindigkeit, und somit großer Energie der Starhlung, ist es jedoch wahrscheinlicher, dass es zu keiner Wechselwirkung kommt, da die Zeit, in der sich der Heliumkern in Wechselwirkungsnähe befindet, geringer ist.\\
Bei hinreichend großen Energien kann der Energieverlust pro Wegstück über die Bethe-Bloch-Gleichung,
\begin{equation}
  -\frac{dE_{\alpha}}{dx} = \frac{z²e^4}{4 \pi \epsilon_0 m_e} \frac{nZ}{v²} \ln{\frac{2m_e v²}{I}},
\end{equation}
bestimmt werden, wobei $z$ die Ladung, $v$ die Geschwindigkeit, $Z$ die Ordnungszahl, $n$ die Teilchendichte und $I$ die Ionisationsenergie des Targetgases ist.
Sie ignoriert jedoch quantenmechanische Effekte wie zum Beispiel den Kernspin.\\
Die Reichweite ergibt sich dann über das Integral
\begin{equation}
  R = \int_0^{E_{\alpha}} -\frac{dE_{\alpha}}{dx} dE_{\alpha}.
\end{equation}
Da bei geringen Energien vermehrt Ladungsaustauschprozesse stattfinden, verliert die Bethe-Bloch-Gleichung ihre Gültigkeit und somit wird die mittlere Reichweite über empirisch gewonnene Kurven bestimmt.
Für sie gilt für $\alpha$-Strahlung in Luft die Gleichung
\begin{equation}
  R_m = \num{3.1} \cdot E_{\alpha}^{3/2},
\end{equation}
wobei die Energie in Megaelektronenvolt angegeben in einem Bereich unter $\SI{2,5}{\mega\electronvolt}$ liegen sollte und $R_m$ in Millimeter angegeben wird.
Unter konstanter Temperatur und konstantem Volumen ist die Reichweite proportional zum umgebenden Druck $p$.
Folglich kann eine zur Ermittlung der Reichweite eine Absorptionsmessung gemacht werden, bei der der Druck variiert wird.
Dadurch gilt für einen festen Abstand $x_0$ zwischen Detektor und Strahler die effektive Weglänge
\begin{equation}
  x_{\text{eff}} = x_0 \frac{p}{p_0},
  \label{eqn:xeff}
\end{equation}
welche durch den Normaldruck $p_0 = \SI{1013}{\milli\bar}$ beschrieben wird. \cite{sample}

% 2x2 Plot
% \begin{figure*}
%     \centering
%     \begin{subfigure}[b]{0.475\textwidth}
%         \centering
%         \includegraphics[width=\textwidth]{Abbildungen/Schaltung1.pdf}
%         \caption[]%
%         {{\small Schaltung 1.}}
%         \label{fig:Schaltung1}
%     \end{subfigure}
%     \hfill
%     \begin{subfigure}[b]{0.475\textwidth}
%         \centering
%         \includegraphics[width=\textwidth]{Abbildungen/Schaltung2.pdf}
%         \caption[]%
%         {{\small Schaltung 2.}}
%         \label{fig:Schaltung2}
%     \end{subfigure}
%     \vskip\baselineskip
%     \begin{subfigure}[b]{0.475\textwidth}
%         \centering
%         \includegraphics[width=\textwidth]{Abbildungen/Schaltung4.pdf}    % Zahlen vertauscht ... -.-
%         \caption[]%
%         {{\small Schaltung 3.}}
%         \label{fig:Schaltung3}
%     \end{subfigure}
%     \quad
%     \begin{subfigure}[b]{0.475\textwidth}
%         \centering
%         \includegraphics[width=\textwidth]{Abbildungen/Schaltung3.pdf}
%         \caption[]%
%         {{\small Schaltung 4.}}
%         \label{fig:Schaltung4}
%     \end{subfigure}
%     \caption[]
%     {Ersatzschaltbilder der verschiedenen Teilaufgaben.}
%     \label{fig:Schaltungen}
% \end{figure*}
