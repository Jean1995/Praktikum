\section{Durchführung}
\label{sec:Durchführung}

Zunächst wird im Versuchsaufbau eine Monozelle betrachtet, dessen Leerlaufspannung $U_0$ gemessen wird.
Dazu wird die Monozelle direkt an ein Voltmeter mit möglichst hohem Eigenwiderstand $R_{\text{Eigen}}$ angeschlossen und die Spannung abgelesen.
Der Eigenwiderstand wird direkt am Voltmeter abgelesen. \\
Als nächstes wird ein einstellbarer Widerstand $R_a$, wie in Abbildung \ref{fig:schaltung1} beschrieben, in die Schaltung eingebaut und im Bereich von $\SI{0}{\ohm}$ bis $\SI{50}{\ohm}$ variiert.
Dabei wird jeweils der Strom $I$ durch ein in Reihe geschaltetes Strommessgerät sowie die Klemmspannung $U_k$ durch ein parallel geschaltetes Spannungsmessgerät betrachtet.
Der Widerstand $R_a$ wird variiert und mehrere Messungen für verschiedene Widerstände durchgeführt.

\begin{figure}[H]
  \centering
  \begin{tikzpicture}[circuit ee IEC, font=\sffamily\footnotesize]
      \draw (0,0) to [battery={name={Bat1}}] (0,3);
      \node at ([xshift=-1em, yshift=+0.5em]Bat1.east) {+};
      \draw (0,3) to (2,3);
      \draw[fill=black] (2,3) circle (1.5pt);
    %  \draw (2,3) to  [Uk={info'={}}](2,0);
      \draw (2,0) to  [voltmeter={info={[yshift=-1.3em, xshift=-1.4em]$U_k$}}](2,3);
      \draw (2,3) to  [ammeter={info'={$I$}}](5,3);
      \draw (5,3) to  [resistor={adjustable={info={$R_a$}}, info'={$\mathsf{_{}}$}}] (5,0);
      \draw (5,0) to (0,0);
      \draw[fill=black] (2,0) circle (1.5pt);
    \end{tikzpicture}
    \caption{Messung von $U_k$ in Abhängigkeit von $R_a$.}
    \label{fig:schaltung1}
\end{figure}

Im folgenden Versuchsteil wird der vorherige Aufbau durch eine Gegenspannung ergänzt, die, wie in Abbildung \ref{fig:schaltung2} beschrieben, hinter den Widerstand in Reihe geschaltet wird.
Die Gegenspannung $U_{\text{Gegen}}$ wird so gewählt, dass sie in etwa $\SI{2}{\volt}$ größer ist als die Leerlaufspannung der Monozelle.
Der Messprozess von $I$ und $U_k$ findet analog zum ersten Aufbau statt.


\tikzset{circuit declare symbol = DC source}
\tikzset{set DC source graphic ={draw,generic circle IEC, minimum size=5mm,info=center:$\stackrel{}{=}$}}

\begin{figure}[H]
  \centering
  \begin{tikzpicture}[circuit ee IEC, font=\sffamily\footnotesize]
      \draw (0,0) to [battery={name={Bat1}}] (0,3);
      \node at ([xshift=-1em, yshift=+0.5em]Bat1.east) {+};
      \draw (0,3) to (2,3);
      \draw[fill=black] (2,3) circle (1.5pt);
    %  \draw (2,3) to  [Uk={info'={}}](2,0);
      \draw (2,0) to  [voltmeter={info={[yshift=-1.3em, xshift=-1.4em]$U_k$}}](2,3);
      \draw (2,3) to  [ammeter={info'={$I$}}](5,3);
      \draw (5,3) to  [resistor={adjustable={info={$R_a$}}, info'={$\mathsf{_{}}$}}] (5,1);
      \draw (5,1) to [DC source={name=Quelle}] (5,0);
      \node at ([shift={(-5pt, 4pt)}]Quelle.north){+};
      \draw (5,0) to (0,0);
      \draw[fill=black] (2,0) circle (1.5pt);
    \end{tikzpicture}
    \caption{Messung von $U_k$ in Abhängigkeit von $R_a$ mit Gegenspannung.}
    \label{fig:schaltung2}
\end{figure}

Zum Schluss wird die Gegenspannung wieder entfernt und die Monozelle als Spannungsquelle, wie in Abbildung \ref{fig:schaltung3} beschrieben, durch einen RC-Generator ersetzt.
Dabei wird einmal eine Messreihe analog zu den bisherigen Messungen mit einer Rechteckspannung sowie einmal mit einer Sinusspannung durchgeführt.
Bei der Messung unter einer Rechteckspannung wird $R_a$ zwischen $\SI{20}{\ohm}$ bis $\SI{250}{\ohm}$ varriert, bei der Messung unter der Sinusspannung zwischen $\SI{100}{\ohm}$ bis $\SI{5000}{\ohm}$.
%Haben wir das überhaupt gemacht, oder haben wir beide male Zwischen 20 und 250 Ohm variiert?

\begin{figure}[H]
  \centering
  \begin{tikzpicture}[circuit ee IEC, font=\sffamily\footnotesize]
      \draw (0,0) to [AC source={info={[yshift=-1.3em, xshift=-2.0em]{$U_0 \, / \, \nu$}},  point down}](0,3);
      \draw (0,3) to (2,3);
      \draw[fill=black] (2,3) circle (1.5pt);
    %  \draw (2,3) to  [Uk={info'={}}](2,0);
      \draw (2,0) to  [voltmeter={info={[yshift=-1.3em, xshift=-1.4em]$U_k$}}](2,3);
      \draw (2,3) to  [ammeter={info'={$I$}}](5,3);
      \draw (5,3) to  [resistor={adjustable={info={$R_a$}}, info'={$\mathsf{_{}}$}}] (5,0);
      \draw (5,0) to (0,0);
      \draw[fill=black] (2,0) circle (1.5pt);
    \end{tikzpicture}
    \caption{Messung von $U_k$ in Abhängigkeit von $R_a$ mit Wechselstromgenerator.}
    \label{fig:schaltung3}
\end{figure}
