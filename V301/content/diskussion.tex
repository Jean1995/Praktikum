\section{Diskussion}
\label{sec:Diskussion}
Im Vergleich von dem direkt gemessenen $U_0$ und den im Nachhinein ermittelten Leerlaufspannungen zeigt sich, besonders bei der Sinusmessung sowie der Rechteckmessung keine signifikante Abweichung.
Bei der Monozellenmessung sowie der Gegenspannungsmessung ergeben sich relative Abweichungen zum direkt gemessenen Wert von
\begin{align*}
f_{\text{mono}} &= \SI{3.87}{\percent}, \\
f_{\text{gegen}} &= \SI{5.49}{\percent}.
\end{align*}
Es fällt auf, dass der direkt gemessene Wert jeweils leicht unterhalb des bestimmten Fehlerintervalls liegt.
Dies mag in dem systematischen Fehler liegen, dass die Nutzung des regelbaren Widerstandes problematisch war.
So zeigt sich, dass die zu ablesenden Werte abhängig vom Druck auf den Inbetriebnahmeschalter des Widerstandes sind.
Dementsprechend gestaltet es sich als schwierig, die Werte $I$ und $U_k$ simultan bei gleichem Widerstand zu bestimmen. \\
Die Berechnung des systematischen Fehlers von $U_0$ zeigt, dass tatsächlich die Annahme $U_0 \approx U_k$ getroffen werden kann, da die Abweichung $\increment{U_0}$ hinreichend klein im Gegensatz zu den anderen genannten systematischen Fehlern ist.\\
Bei Betrachtung der Leistungskurve ist beim Peak der Theoriekurve eine nach oben tendierende Abweichung der Messwerte im Vergleich zur Theoriekurve zu beobachten.
Diese Ursache dieser Abweichung kann nicht genau zurückverfolgt werden.
Ansonten liegen die Werte der Theoriekurve im Rahmen der Fehlerbalken der Messwerte.
