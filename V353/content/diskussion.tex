\section{Diskussion}
\label{sec:Diskussion}
Die Werte der Zeitkonstante unterscheiden sich jeweils um etwa $\SI{0.1}{\milli\second}$, was jeweils weit oberhalb der jeweiligen Fehler liegt.
Zunächst zur Bestimmung mithilfe der Aufladekurve:
Diese Methode erweist sich als nicht sehr zuverlässig, da es schwierig ist, gewisse Wertepaare sicher abzulesen.
Zuverlässiger mag die Bestimmung nach mithilfe der Frequenzabhängigkeit der Spannungsamplitude zu sein, wobei es auch nicht einfach ist, mit einem digitalen Oszilloskop die Amplitudenwerte exakt abzumessen.
Die deutliche Verbesserung zu der Messung anhand der Aufladekurve ist jedoch, dass der Innenwiderstand der Spannungsquelle berücksichtigt wird, was einen systematischen Fehler eliminiert.
Bei der Messung anhand der frequenzabhängigen Phasenverschiebung, zeigt sich eine signifikante Abweichung der Messwerte gegenüber der Ausgleichskurve bei hohen Frequenzen.
Dies resultiert in einem fünf mal größeren Fehler als bei der Spannungsamplitudenmessung, trotz gleicher Anzahl an Messwerten.
Generell trifft der systematische Fehler auf, dass ein zu hoher Frequenzbereich untersucht wurde.
Weitere Messungen in niedrigeren Frequenzbereichen hätten die Messergebnisse bereichert.
Insgesamt liegen die Messwerte recht gut auf den jeweilig erwarteten Funktionen.
Die Methode nach mithilfe der Spannungsamplitude scheint am zuverlässigsten zu sein, da sie gewisse systematische Fehler eliminiert und den geringsten Fehler innehält.
