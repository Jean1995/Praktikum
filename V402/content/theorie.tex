\section{Theorie}
\label{sec:Theorie}

% 2x2 Plot
% \begin{figure*}
%     \centering
%     \begin{subfigure}[b]{0.475\textwidth}
%         \centering
%         \includegraphics[width=\textwidth]{Abbildungen/Schaltung1.pdf}
%         \caption[]%
%         {{\small Schaltung 1.}}
%         \label{fig:Schaltung1}
%     \end{subfigure}
%     \hfill
%     \begin{subfigure}[b]{0.475\textwidth}
%         \centering
%         \includegraphics[width=\textwidth]{Abbildungen/Schaltung2.pdf}
%         \caption[]%
%         {{\small Schaltung 2.}}
%         \label{fig:Schaltung2}
%     \end{subfigure}
%     \vskip\baselineskip
%     \begin{subfigure}[b]{0.475\textwidth}
%         \centering
%         \includegraphics[width=\textwidth]{Abbildungen/Schaltung4.pdf}    % Zahlen vertauscht ... -.-
%         \caption[]%
%         {{\small Schaltung 3.}}
%         \label{fig:Schaltung3}
%     \end{subfigure}
%     \quad
%     \begin{subfigure}[b]{0.475\textwidth}
%         \centering
%         \includegraphics[width=\textwidth]{Abbildungen/Schaltung3.pdf}
%         \caption[]%
%         {{\small Schaltung 4.}}
%         \label{fig:Schaltung4}
%     \end{subfigure}
%     \caption[]
%     {Ersatzschaltbilder der verschiedenen Teilaufgaben.}
%     \label{fig:Schaltungen}
% \end{figure*}

Als Dispersion wird in der Optik die Abhängigkeit des Brechungsindex $n$ von der Wellenlänge des Lichts bezeichnet.
Dabei ist der Brechungsindex definiert als
\begin{equation}
  n = \frac{v_1}{v_2},
\end{equation}
also als Quotient der Ausbreitungsgeschwindigkeiten in zwei angrenzenden Medien.
Das Verhalten eines Lichtsstrahles an einer solchen Grenzfläche kann unter Verwendung des Huygenschen Prinzps beschrieben werden.
Dieses besagt, dass von jeden Punkt einer Wellenfront eine neue, kugelförmige Welle ausgeht.
Aus elementargeometrischen Überlegungen folgt hieraus, dass das Snelliussche Brechungsgesetz
\begin{equation}
  \frac{\sin{\alpha}}{\sin{\beta}} = \frac{v_1}{v_2} = n
  \label{eqn:snel}
\end{equation}
gilt.
Hierbei beschreibt $\alpha$ den Winkel zwischen dem Lot der Grenzfläche und dem einfallenden Strahl sowie $\beta$ den Winkel zwischen dem Lot der Grenzfläche und dem transmittierten Strahl.
Zudem beschreiben $v_1$ und $v_2$ die dort vorhandenen Ausbreitungsgeschwindigkeiten.\\
Im Allgemeinen ist der Brechungsindex $n(\lambda)$ jedoch, wie oben bereits erwähnt, eine komplexe, von der Wellenlänge abhängige Größe.
Um diesen Zusammenhang herzuleiten, wird die Wechselwirkung zwischen dem elektrischen Feldes des Lichts und den Ladungen in der Materie betrachtet.
Sie wird durch eine lineare, nicht homogene Differentialgleichung zweiter Ordnung beschrieben, bei der die Kraft des E-Feldes auf die Ladungen, eine lineare rücktreibende Kraft $( \propto a_h )$ durch die Auslenkung aus der Ruhelage sowie ein geschwindigkeitsabhängige Reibungsterm $(\propto f_h)$ berücksichtigt werden.
Die Lösung dieser Differentialgleichung ergibt für die Polarisation $\vec{P_h}$ den Ausdruck
\begin{equation}
  \vec{P_h} = \frac{1}{\omega_h^2 - \omega^2 + i \frac{f_h}{m_h}\omega} \frac{N_q q_h^2}{m_h} \vec{E_0} \exp{( i \omega t )}
\end{equation}
beschrieben.
Dabei steht $m_h$ für die Teilchenmasse, $q_h$ für die Teilchenladung, $N_h$ für die Teilchenanzahl pro Volumeneinheit und $\omega_h^2 = \frac{a_h}{m_h}$ für die Resonanzfrequenz des Systems.
Unter Verwendung des Zusammenhangs
\begin{equation}
  \vec{P} = (\epsilon - 1) \epsilon_0 \vec{E}
\end{equation}
mit der Dielektrizitätszahl $\epsilon$ sowie der Maxwellschen Relation
\begin{equation}
  n^2 = \epsilon
\end{equation}
kann der im Allgemeinen komplexe Ausdruck $\tilde{n}$ erhalten werden.
Dessen Imaginärteil beschreibt die Dämpfung der Lichts, welche jedoch für den hier betrachteten Bereich außerhalb der Resonanzfrequenz vernachlässigt werden kann.
In diesem Fall vereinfacht sich der Realteil von $\tilde{n}$, welcher den oben beschriebenen geometrischen Zusammenhang beschreibt, zu
\begin{equation}
  n^2(\lambda) = 1 + \sum \frac{N_h q_h^2}{4 \pi^2 c^2 \epsilon_0 m_h} \frac{\lambda^2 \lambda_h^2}{\lambda^2 - \lambda_h^2}. \label{kakapipi_ja_das_habe_ich_wirklich_geschrieben}
\end{equation}
Unter der Annahme, dass nur eine Absorptionsstelle $\lambda_1$ existiert, kann nun zwischen zwei Fällen unterschieden werden:
Bei der Betrachtung von Wellenlängen $\lambda \gg \lambda_1$ kann die Dispersionsrelation als Darstellung
\begin{equation}
  n^2(\lambda) = A_0 + \frac{A_2}{\lambda^2} + \frac{A_4}{\lambda^4} + \dotsb
  \label{eqn:d1}
\end{equation}
entwickelt werden.
Diese Entwicklung entspricht einer positiven Krümmung der Dispersionskurve, das heißt einer Linkskrümmung.
Im anderen Fall, also $\lambda \ll \lambda_1$, kann die Entwicklung der Dispersionsrelation als
\begin{equation}
  n^2(\lambda) =  1 - A_2 \lambda^2 - A_4 \lambda^4 - \dotsb
  \label{eqn:d2}
\end{equation}
geschrieben werden.
Hier wird also von einer negativen Krümmung, also einer nach rechts gekrümmten Dispersionskurve ausgegangen.
