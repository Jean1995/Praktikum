\section{Diskussion}
\label{sec:Diskussion}
Der bestimmte Wert für den Winkel des Prismas lautet
\begin{align*}
  \varphi &= \input{build/phi.tex}.
\end{align*}
Dieser könnte vom wahren Wert weit abweichen, da das verwendete Prisma angebrochene Kanten hat.
Um Folgefehler zu vermeiden wurde für die darauffolgenden Rechnungen nicht der gemessene Wert, sondern der Wert $\varphi = \SI{60}{\degree}$ verwendet.\\
Die Dispersionskurve wird nach der ersten Methode \eqref{eqn:d1} mit einem geringen Fehler und den Parametern
\begin{align*}
A_0 &= \input{build/A_0.tex},\\
A_2 &= \input{build/A_2.tex},\\
A_4 &= \input{build/A_4.tex}
\end{align*}
angenähert.\\
Für die Abbesche Zahl ergibt sich ein Wert von
\begin{align*}
  \nu &= \input{build/abbesche_zahl.tex},
\end{align*}
welcher auf Flintglas schließen lässt.
Es wird gekennzeichnet mit der Eigenschaft einen hohen Brechungsindex zu haben.\\
Des Weiteren beträgt die Auflösung der beiden Fraunhoferlinien
\begin{align*}
A_C &= \input{build/A_lambda_c.tex},\\
A_F &= \input{build/A_lambda_f.tex}.
\end{align*}
Die Absorptionslinie befindet sich laut Berechnung bei einer Wellenlänge von
\begin{align*}
  \lambda_i &= \input{build/lambda_i.tex}.
\end{align*}
Da sie im UV-Bereich liegt, sollte sie richtig bestimmt worden sein.
Jedoch handelt es sich hierbei nur um eine Näherung, durchgeführt mit gefitteten Parametern, welche auf einem kleinen Messintervall basieren.
Daher sollte die Gültigkeit des Wertes nicht zu hoch eingeschätzt werden.


%Wert für Absorptionsstelle scheint logisch, weil sie im UV-Bereich liegt und das Prisma durchsichtig ist
