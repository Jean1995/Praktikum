\documentclass[
  bibliography=totoc,     % Literatur im Inhaltsverzeichnis
  captions=tableheading,  % Tabellenüberschriften
  titlepage=firstiscover, % Titelseite ist Deckblatt
]{scrartcl}
\usepackage{parskip}
% Paket float verbessern
\usepackage{scrhack}

% Warnung, falls nochmal kompiliert werden muss
\usepackage[aux]{rerunfilecheck}

% deutsche Spracheinstellungen
\usepackage{polyglossia}
\setmainlanguage{german}

% unverzichtbare Mathe-Befehle
\usepackage{amsmath}
% viele Mathe-Symbole
\usepackage{amssymb}
% Erweiterungen für amsmath
\usepackage{mathtools}

% Fonteinstellungen
\usepackage{fontspec}
% Latin Modern Fonts werden automatisch geladen

\usepackage[
  math-style=ISO,    % ┐
  bold-style=ISO,    % │
  sans-style=italic, % │ ISO-Standard folgen
  nabla=upright,     % │
  partial=upright,   % ┘
  warnings-off={           % ┐
    mathtools-colon,       % │ unnötige Warnungen ausschalten
    mathtools-overbracket, % │
  },                       % ┘
]{unicode-math}

% traditionelle Fonts für Mathematik
\setmathfont{Latin Modern Math}
\setmathfont{XITS Math}[range={scr, bfscr}]
\setmathfont{XITS Math}[range={cal, bfcal}, StylisticSet=1]

% Zahlen und Einheiten
\usepackage[
  locale=DE,                 % deutsche Einstellungen
  separate-uncertainty=true, % immer Fehler mit \pm
  per-mode=reciprocal,       % ^-1 für inverse Einheiten
  output-decimal-marker=.,   % . statt , für Dezimalzahlen
]{siunitx}

% chemische Formeln
\usepackage[
  version=4,
  math-greek=default, % ┐ mit unicode-math zusammenarbeiten
  text-greek=default, % ┘
]{mhchem}

% richtige Anführungszeichen
\usepackage[autostyle]{csquotes}

% schöne Brüche im Text
\usepackage{xfrac}

% Standardplatzierung für Floats einstellen
\usepackage{float}
\floatplacement{figure}{htbp}
\floatplacement{table}{htbp}

% Floats innerhalb einer Section halten
\usepackage[
  section, % Floats innerhalb der Section halten
  below,   % unterhalb der Section aber auf der selben Seite ist ok
]{placeins}

% Seite drehen für breite Tabellen
\usepackage{pdflscape}

% Captions schöner machen.
\usepackage[
  labelfont=bf,        % Tabelle x: Abbildung y: ist jetzt fett
  font=small,          % Schrift etwas kleiner als Dokument
  width=0.9\textwidth, % maximale Breite einer Caption schmaler
]{caption}
% subfigure, subtable, subref
\usepackage{subcaption}

% Grafiken können eingebunden werden
\usepackage{graphicx}
% größere Variation von Dateinamen möglich
\usepackage{grffile}

% schöne Tabellen
\usepackage{booktabs}

% Verbesserungen am Schriftbild
\usepackage{microtype}

% Literaturverzeichnis
\usepackage[
  style=numeric,
  sorting=none,
  backend=biber,
]{biblatex}
% Quellendatenbank
\addbibresource{lit.bib}
\addbibresource{programme.bib}

%Biber bringt nichts durcheinander
%\usepackage[sort&compress,numbers]{natbib}

% Hyperlinks im Dokument
\usepackage[
  unicode,        % Unicode in PDF-Attributen erlauben
  pdfusetitle,    % Titel, Autoren und Datum als PDF-Attribute
  pdfcreator={},  % ┐ PDF-Attribute säubern
  pdfproducer={}, % ┘
]{hyperref}
% erweiterte Bookmarks im PDF
\usepackage{bookmark}

% Trennung von Wörtern mit Strichen
\usepackage[shortcuts]{extdash}

% tikzpicture
\usepackage{tikz}
\usetikzlibrary{circuits.ee.IEC}
\usetikzlibrary{positioning}
\tikzset{
  Pfeil/.style={thick,shorten >=#1,shorten <=#1,->,>=latex}, % für Peile
  UPfeil/.style={blue,Pfeil=#1,font={\sffamily\itshape}},% für Spannungspfeile
  IPfeil/.style={red,Pfeil=#1,font={\ttfamily\itshape}} % für Strompfeile
}
%Volt- und Amperemeter festlegen:
\tikzset{circuit declare symbol = Us}
\tikzset{set Us graphic ={draw,generic circle IEC, minimum size=5mm,info=center:Us}}

\tikzset{circuit declare symbol = voltmeter}
\tikzset{set voltmeter graphic ={draw,generic circle IEC, minimum size=5mm,info=center:V}}

\tikzset{circuit declare symbol = ammeter}
\tikzset{set ammeter graphic ={draw,generic circle IEC, minimum size=5mm,info=center:A}}


\tikzset{circuit declare symbol = AC source}
\tikzset{AC source IEC graphic/.style={
    circuit symbol lines,
    circuit symbol size=width 2 height 2,
    shape=generic circle IEC,
    /pgf/generic circle IEC/before background={
    \pgfpathmoveto{\pgfpoint{-0.8pt}{0pt}}
    \pgfpathsine{\pgfpoint{0.4pt}{0.4pt}}
    \pgfpathcosine{\pgfpoint{0.4pt}{-0.4pt}}
    \pgfpathsine{\pgfpoint{0.4pt}{-0.4pt}}
    \pgfpathcosine{\pgfpoint{0.4pt}{0.4pt}}
    \pgfusepath{stroke}
    },
    transform shape, draw
  }
}
\tikzset{circuit ee IEC/.append style=
  {set AC source graphic = AC source IEC graphic}
}

\author{
  Johannes Kollek%
  \texorpdfstring{
    \\
    \href{mailto:johannes.kollek@udo.edu}{johannes.kollek@udo.edu}
  }{}%
  \texorpdfstring{\and}{, }
  Jean-Marco Alameddine%
  \texorpdfstring{
    \\
    \href{mailto:jean-marco.alameddine@udo.edu}{jean-marco.alameddine@udo.edu}
  }{}%
}
\publishers{TU Dortmund – Fakultät Physik}


%\subject{Versuchsprotokoll zum Versuch Nr. XXX}
%\title{XXX}
%\date{
%  Durchführung: xx.xx.2015
%}

\begin{document}

%\maketitle
%\thispagestyle{empty}
%\newpage

%\section{Versuchsziel}
\label{sec:Versuchsziel}
Das Ziel des Experiments ist es, die Wellenlänge eines Lasers und die Brechungsindizes zweier Gase mit Hilfe eines Michelson-Interferometers zu bestimmen.

\section{Theorie}
\label{sec:Theorie}

In letzter Zeit ist es Wissenschaftlern gelungen, Gravitationswellen nachzuweisen.
Hierzu wurde ein Michelson-Interferometer benutzt, da es dazu geeignet ist, Wellenlängen, sowie deren absolute Änderung zu messen.
Zusätzlich können auch Brechungsindizes gemessen werden.
Das Michelson-Interferometer selbst stützt sich auf das Interferenzprinzip, auf welches im Folgenden weiter eingegangen wird.\\
\subsection{Interferenz des Lichtes}
Im Allgemeinen kann Licht als elektromagnetische Welle der Form
\begin{equation}
  \vec{E}(x,t) = \vec{E_0}\exp{(i(\vec{k}\vec{x}-\omega t -\delta))}, \label{eqn:1}
\end{equation}
mit Ort $\vec{x}$ und Zeit $t$, sowie Wellenvektor $\vec{k}$, Frequenz $\omega$ und Phasenkonstante $\delta$, angenommen werden, wobei sie sich nur in ihrer Intensität
\begin{equation}
  I = const |\vec{E}|²
\end{equation}
messbar macht.
Da diese elektromagnetischen Wellen den Maxwell-Gleichungen unterliegen, gilt die lineare Superposition, wodurch sich zwei Wellen einfach addieren lassen.
Aus Gründen der Messbarkeit muss hier jedoch auf ein zeitliches Mittel zurückgegriffen werden, wodurch sich für die gesamte Intensität $I_{\text{ges}}$ der Ausdruck
\begin{equation}
  I_{\text{ges}} = \frac{const}{t_2-t_1} \int_{t_1}^{t_2} (|\vec{E_1}+\vec{E_2}|)²(x,t) dt
\end{equation}
ergibt. Dabei muss darauf geachtet werden, dass die Periodendauer $T = 2\pi/\omega$ groß gegenüber der Länge des Messintervalls ist.
Somit ergibt sich für zwei Lichtwellen der Form \eqref{eqn:1}, welche sich nur in ihrer Phase unterscheiden,
\begin{equation}
  I_{\text{ges}} = const \cdot 2\vec{E_0}(1+\cos{\delta_2-\delta_1}).
\end{equation}
Hierbei spiegelt der Cosinus den Interferenzterm wieder.
Erkennbar ist, dass, wenn der Phasenunterschied $(\delta_2-\delta_1)$ ungerade Vielfache von $\pi$ beträgt, es zur Auslöschung der beobachtbaren Intensität führt.
Dies wird als destruktive Interferenz bezeichnet.
Ein Kernkriterium, unter dem Interferenzerscheinungen beobachtbar werden, ist die Kohärenz der beiden Lichtbündel.\\

\subsection{Kohärenz und Kohärenzlänge}
Zwei Wellen sind zueinander zeitlich kohärent, wenn sie, abgesehen von ihrer Phase, zeitlich das selbe Schwingungsmuster in ihrem Zug aufweisen.
Dabei ist auf die Entstehung des Lichtes zu achten:
Licht wird in einem gewissen Zeitfenster von angeregten Atomen emittiert.
Bei mehreren Atomen (im Prinzip Lichtquellen) überlagern sich ihre Wellengruppen.
Hierbei ist zu erwähnen, dass ihre Phasenkonstanten im Allgemeinen statistische Funktionen $\delta(t)$ der Zeit sind.
Demnach können verschieden große bzw. kleine Werte angenommen werden.
Daraus folgt, dass der gemittelte Cosinus der Phasendifferenz,
\begin{equation}
  \frac{1}{t_2-t_1} \int_{t_1}^{t_2} \cos{\delta_2(t)-\delta_1(t)}dt \approx 0,
\end{equation}
nicht mehr beobachtbar ist.
Deswegen ist das Licht zweier unterschiedlicher Lichtquellen allgemein nicht interferenzfähig; es ist inkohärent.\\
Prinzipiell kann das Licht einer einzigen Lichtquelle aufgespalten und wieder zusammengeführt werden, um Interferenzeffekte zu beobachten.
Dabei wird bevorzugt das Licht von Lasern verwendet, da es mit ihnen möglich ist kohärentes Licht zu erzeugen.
Dies ist schematisch in Abbildung \ref{abb:1} dargestellt.
\begin{figure}[H]
  \centering
  \includegraphics[height=4cm]{ressources/spaltung.png}
  \caption{Aufspaltung und Zusammenführung eines Lichtbündels. \cite{Quelle0}}
  \label{abb:1}
\end{figure}
Beträgt die Differenz der beiden Wege nun ungeradzahlige Vielfache der halben Wellenlänge,
\begin{equation}
  \Delta = (2n+1)\frac{\lambda}{2},
\end{equation}
so kommt es erneut zu Auslöschungseffekten am Punkt P.
Wichtig ist zusätzlich noch die Kohärenzlänge $l$.
Da, wie bereits erwähnt, die Emission von Licht in endlichen Zeitabständen $\tau$ stattfindet, haben die Wellenzüge eine endliche Länge.
Wird diese überschritten, überlagern sich am Zusammentreffungspunkt zwei Wellenzüge von unterschiedlichen Emissionsakten, was bei hinreichend großem Wegunterschied zum Verschwinden des Interferenzeffektes führt.
Demnach wird die Kohärenzlänge festgelegt als das Produkt der bei P maximal beobachtbaren Intensitätsmaxima $N$ mit der Wellenlänge $\lambda$ des Lichtes,
\begin{equation}
  l = N\lambda.\\
\end{equation}
Zusätzlich kann die Breite der Lichtquelle zur Verhinderung der Interferenzerscheinungen führen, da das emittierte Licht an verschiedenen Stellen möglicherweise Phasenverschiebungen aufweist, die insgesamt in einer Inkohärenz enden.\\


\subsection{Wellenlängenmessung mittels Interferenz}

Die Quintessenz dieses Abschnittes lautet, dass sich die Wellenlänge einer Lichtquelle über eine Aufspaltung des Lichtes auf verschieden lange Wege, wobei einer um eine Strecke $\Delta l$ variiert wird, und anschließender Zusammenführung der Strahlen durch eine Zählung der entstehenden Interferenzmaxima bestimmen lässt.
Dies genügt der Gleichung
\begin{equation}
  \Delta l = N\frac{\lambda}{2}, \label{eqn:2}
\end{equation}
bei der $N$ die gezählten Interferenzmaxima beschreibt.\\

\subsection{Bestimmung eines Brechungsindex über Interferenz}

Befindet sich die Messapparatur in einem Medium, herrscht auf jeder Wegstrecke der Brechungsindex $n$.
Wird ein anderes Medium der Dicke $b$ einem Strahlengang in den Weg gesetzt, durchläuft das Licht auf diesem Ast eine zusätzliche effektive Weglänge von $b\Delta n$.
Dadurch lässt sich bei bekannter Wellenlänge die Änderung des Brechungsindex $\Delta n$,
\begin{equation}
  b\Delta n = N\frac{\lambda}{2}, \label{eqn:3}
\end{equation}
beschreiben.
Zur Berechnung des Brechungsindex unter Normalbedingungen wird die Formel
\begin{equation}
  n(p_0,T_0) = 1 +\Delta n(\Delta p) \frac{T}{T_0}\frac{p_0}{\Delta p}
\end{equation}
verwendet, bei der $p_0$ den Normaldruck und $T_0$ die Normaltemperatur beschreibt.
Wird diese Formel durch den Ausdruck (\ref{eqn:3}) ergänzt ergibt sich
\begin{equation}
  n(p_0,T_0) = 1 +\frac{N \lambda}{2b} \frac{T}{T_0}\frac{p_0}{\Delta p}, \label{eqn:soso}
\end{equation}
wobei $\lambda$ die Wellenlänge des Lasers und $b$ die Länge des Gasbehälters ist, welchen das Laserlicht passiert.





% 2x2 Plot
% \begin{figure*}
%     \centering
%     \begin{subfigure}[b]{0.475\textwidth}
%         \centering
%         \includegraphics[width=\textwidth]{Abbildungen/Schaltung1.pdf}
%         \caption[]%
%         {{\small Schaltung 1.}}
%         \label{fig:Schaltung1}
%     \end{subfigure}
%     \hfill
%     \begin{subfigure}[b]{0.475\textwidth}
%         \centering
%         \includegraphics[width=\textwidth]{Abbildungen/Schaltung2.pdf}
%         \caption[]%
%         {{\small Schaltung 2.}}
%         \label{fig:Schaltung2}
%     \end{subfigure}
%     \vskip\baselineskip
%     \begin{subfigure}[b]{0.475\textwidth}
%         \centering
%         \includegraphics[width=\textwidth]{Abbildungen/Schaltung4.pdf}    % Zahlen vertauscht ... -.-
%         \caption[]%
%         {{\small Schaltung 3.}}
%         \label{fig:Schaltung3}
%     \end{subfigure}
%     \quad
%     \begin{subfigure}[b]{0.475\textwidth}
%         \centering
%         \includegraphics[width=\textwidth]{Abbildungen/Schaltung3.pdf}
%         \caption[]%
%         {{\small Schaltung 4.}}
%         \label{fig:Schaltung4}
%     \end{subfigure}
%     \caption[]
%     {Ersatzschaltbilder der verschiedenen Teilaufgaben.}
%     \label{fig:Schaltungen}
% \end{figure*}

%\subsection{Durchführung}
\label{sec:durchführung}

%\section{Auswertung}
\label{sec:Auswertung}

Die Materialeigenschaften der auf der Grundplatte verbauten Stäbe werden der Versuchsdurchführung entnommen und in Tabelle \ref{tab:1} angegeben.

\begin{table}
  \centering
  \caption{Materialeigenschaften der Grundplatte. \cite{sample}}
  \label{tab:1}
  \sisetup{table-format=1.2}
  \begin{tabular}{c c c c c c}
    \toprule
    {$\text{Material}$} & {$l \:/\: \si{\centi\metre}$} & {$b \:/\: \si{\centi\metre}$} & {$h \:/\: \si{\centi\metre}$} & {$\rho \:/\: \si{\kilo\gram\per\metre\tothe{3}}$} & {$c \:/\: \si{\joule\per\kilo\gram\per\kelvin} $} \\
    \midrule
    Messing  & 9 & 1.2 & 0.4 & 8520 & 385 \\
    Messing  & 9 & 0.9 & 0.4 & 8520 & 385 \\
    Aluminium  & 9 & 1.2 & 0.4 & 2800 & 830 \\
    Edelstahl  & 9 & 1.2 & 0.4 & 8000 & 400 \\

    \bottomrule
  \end{tabular}
\end{table}

Zunächst wird die erste Messung, wie in Kapitel \ref{sec:stat} beschrieben, durchgeführt.


\subsection{Qualitative Beschreibung des Temperaturverlaufes}


In Abbildung \ref{fig:1} wird der zeitliche Verlauf für $T_1$ und $T_4$ dargestellt, in \ref{fig:2} der zeitliche Verlauf von $T_5$ und $T_8$.

\begin{figure}
  \centering
  \includegraphics[height=13cm, angle=270]{scan-7.jpg}
  \caption{Zeitlicher Temperaturverlauf von $T_1$ und $T_4$.}
  \label{fig:1}
\end{figure}

\begin{figure}[H]
  \centering
  \includegraphics[height=13cm, angle=270]{scan-6.jpg}
  \caption{Zeitlicher Temperaturverlauf von $T_5$ und $T_8$.}
  \label{fig:2}
\end{figure}

Für Abbildung \ref{fig:1} fällt auf, dass beide Graphen eine ähnliche Form aufweisen.
Nach erst ca. $\SI{200}{\second}$ bildet sich eine Temperaturdifferenz, die nach ungefähr $\SI{500}{\second}$ konstant $\SI{1}{\kelvin}$ beträgt.
Die Graphen wachsen bis zum Ende der Messung in einer identischen Form weiter.
Diese Ähnlichkeiten lassen sich darauf zurückführen, dass es sich um das gleiche Material, nämlich Messing handelt.
Die Unterschiede werden durch die verschiedenen Ausmaße verursacht.\\
Für Abbildung \ref{fig:2} fällt auf, dass der Graph von $T_5$ eine zu Beginn stärkere Steigung aufweist als der Graph von $T_8$.
Nach einiger Zeit stellt sich eine annähernd konstante Temperaturdifferenz von ca. $\SI{6}{\kelvin}$ ein, welche bis zum Ende der Messung bestehen bleibt.
Grund hierfür ist, dass es sich beim ersten Material um Aluminium handelt, welches scheinbar eine bessere Wärmeleitfähigkeit als das andere Material, nämlich Edelstahl, besitzt.\\
Betrachtet man genauer die Temperaturen, welche für die verschiedenen Messpunkte nach $\SI{700}{\second}$ auftreten, so erhält man
\begin{align*}
  T_1 &= \SI{35.70}{\celsius}  \:\: \text{(Messing)}, \\
  T_4 &= \SI{34.73}{\celsius} \:\: \text{(Messing)}, \\
  T_5 &= \SI{37.26}{\celsius} \:\: \text{(Aluminium)}, \\
  T_8 &= \SI{31.14}{\celsius} \:\: \text{(Edelstahl}).
\end{align*}
Diese Werte führen zu der Annahme, dass die beste Wärmeleitfähigkeit bei Aluminium vorliegt, gefolgt von Messing, wobei hier die Wärmeleitfähigkeit von den Maßen abhängt.
Die geringste Wärmeleitfähigkeit kann dem Edelstahl zugeordnet werden.\\


\subsection{Bestimmung des Wärmestroms}


Der Wärmestrom, der zu einem bestimmten Zeitpunkt fließt, kann nach Gleichung \ref{eqn:w} berechnet werden.
Es werden dafür die Temperaturen $T_1$ und $T_2$, also von Messing betrachtet.
Die Temperaturdifferenzen können der Abbildung \ref{fig:3} entnommen werden, die weiteren Größen sind
\begin{align*}
  \increment{x} &= \input{build/x_ws.tex}\si{\metre} \\
  A &= \input{build/a_ws.tex}\si{\metre\tothe{3}} \\
  \kappa_\text{Messing} &= \input{build/k_ws.tex}\si{\watt\per\metre\per\kelvin} \\
\end{align*}
wobei $\increment{x}$, der Abstand der Messstellen, am Aufbau abgelesen wird, der Querschnitt $A$ aus Tabelle \ref{tab:1} errechnet und $\kappa_\text{Messing}$ der Literatur \cite{etb} entnommen wird.
Es ergeben sich somit die in Tabelle \ref{tab:2} angegebenen Wärmeströme für die jeweiligen Messzeiten.

\begin{table}
  \centering
  \caption{Wärmeströme für Messing.}
  \label{tab:2}
  \sisetup{table-format=1.2}
  \begin{tabular}{c c}
    \toprule
    {$t \:/\: \si{\second}$} & {$\frac{\increment{Q}}{\increment{t}} \:/\: \si{\watt}$}\\
    \midrule
    \input{build/wstrom.tex}
    \bottomrule
  \end{tabular}
\end{table}

Zuletzt werden die Temperaturdifferenzen zwischen dem Ende sowie dem Anfang des Stabes gegen die Zeit abgetragen.
Es ergibt sich für Messing der in Abbildung \ref{fig:3} dargestellte Graph, für Edelstahl der in Abbildung \ref{fig:4} dargestellte Graph.
\begin{figure}[H]
  \centering
  \includegraphics[height=12.5cm, angle=270]{scan-4.jpg}
  \caption{Zeitlicher Verlauf der Temperaturdifferenz von $T_2$ zu $T_1$.}
  \label{fig:3}
\end{figure}

\begin{figure}[H]
  \centering
  \includegraphics[height=12.5cm, angle=270]{scan-5.jpg}
  \caption{Zeitlicher Verlauf der Temperaturdifferenz von $T_7$ zu $T_8$.}
  \label{fig:4}
\end{figure}

Beide Graphen zeigen zu Beginn eine Temperaturdifferenz von $\SI{0}{\kelvin}$ auf, welche innerhalb der nächsten ca. $\SI{100}{\second}$ bis auf $\SI{6.5}{\kelvin}$ für Edelstahl bzw. $\SI{3}{\kelvin}$ Kelvin für Messing ansteigt.
Beim Messing ist hier ein starker Peak zu betrachten, im weiteren Verlauf sinkt die Temperaturdifferenz wieder auf ein bis zum Ende der Messung konstantes Niveau von ca. $\SI{1.7}{\kelvin}$ ab.
Im Gegensatz dazu befindet sich beim Edelstahl nach Erreichen des Peaks kein drastischer Abfall der Temperaturdifferenz, die Temperaturdifferenz nimmt nur leicht ab und bleibt auf einer konstanten Temperatur von ca. $\SI{6.3}{\kelvin}$.
Dieses Verhalten ist ein Indiz für die bessere Wärmeleitfähigkeit von Messing, da sich hier ein besserer Temperaturausgleich bei gleicher Erwärmung einstellt.

\subsection{Bestimmung der Wärmeleitfähigkeit mittels Angström-Messverfahren}
\subsubsection{Bestimmung der Wärmeleitfähigkeit für Messing}
Es wird eine Messung mittels dynamischer Methode, wie in Kapitel \ref{sec:dyn} beschrieben, durchgeführt.
Für die Messung ergibt sich der in Abbildung \ref{fig:5} dargestellte Graph.

\begin{figure}[H]
  \centering
  \includegraphics[height=13cm, angle=270]{scan-3.jpg}
  \caption{Zeitlicher Verlauf der Temperaturen für $T_1$ und $T_2$ bei periodischer Erwärmung nach der Angström-Methode.}
  \label{fig:5}
\end{figure}

Aus diesem Graphen können die Amplituden sowie zeitlichen Lagen der Maxima bestimmt werden.
Die Amplituden werden hierbei bestimmt, indem die Amplitudendifferenz von einem Maximum und dem jeweils vorherigen Minimum genommen und halbiert wird.
Die Werte sind in Tabelle \ref{tab:3} angegeben.

\begin{table}
  \centering
  \caption{Maxima der Temperaturen $T_1$ und $T_2$.}
  \label{tab:3}
  \sisetup{table-format=1.2}
  \begin{tabular}{c c c c}
    \toprule
    {$A_1 \:/\: \si{\kelvin}$} & {$A_2 \:/\: \si{\kelvin}$}  & {$t_1 \:/\: \si{\second}$}  & {$t_2 \:/\: \si{\second}$}\\
    \midrule
    \input{build/dyn1.tex}
    \bottomrule
  \end{tabular}
\end{table}

Nach Formel \eqref{eqn:wlf} wird hieraus für jede Amplitude die Wärmeleitfähigkeit bestimmt.
Die Ergebnisse sind in Tabelle \ref{tab:4} dargestellt.

\begin{table}
  \centering
  \caption{Einzelne Ergebnisse der Wärmeleitfähigkeit für Messing.}
  \label{tab:4}
  \sisetup{table-format=1.2}
  \begin{tabular}{c}
    \toprule
    {$\kappa_\text{Messing} \:/\: \si{\watt\per\metre\per\kelvin}$}\\
    \midrule
    \input{build/kappa_1_tab.tex}
    \bottomrule
  \end{tabular}
\end{table}

Diese Werte werden nun gemittelt, wobei sich der Mittelwert nach
\begin{equation}
  \bar{x} = \frac{1}{n} \sum_{i=1}^n x_i
\end{equation}
und die Standardabweichung nach
\begin{equation}
  S = \sqrt{ \frac{1}{n-1} \sum_{i=1}^n  \bigl(x_i - \bar{x} \bigr)^2  }
\end{equation}
berechnet.\\
Hieraus folgt eine mittlere Wärmeleitfähigkeit von
\begin{align*}
  \kappa_\text{Messing} &= \input{build/kappa_1.tex}. \\
\end{align*}
\subsubsection{Bestimmung der Wärmeleitfähigkeit für Aluminium}
Ein analoges Vorgehen wird nun für Aluminium durchgeführt.
Der bei dieser Messung entstandene Graph ist in Abbildung \ref{fig:6} dargestellt.
\begin{figure}[H]
  \centering
  \includegraphics[height=13cm, angle=270]{scan-2.jpg}
  \caption{Zeitlicher Verlauf der Temperaturen für $T_5$ und $T_6$ bei periodischer Erwärmung nach der Angström-Methode.}
  \label{fig:6}
\end{figure}
Für diesen Graphen werden die Amplituden sowie die zeitliche Lage der Amplituden zu der in Tabelle \ref{tab:5} angegebenen Werte bestimmt.
\begin{table}
  \centering
  \caption{Maxima der Temperaturen $T_5$ und $T_6$.}
  \label{tab:5}
  \sisetup{table-format=1.2}
  \begin{tabular}{c c c c}
    \toprule
    {$A_5 \:/\: \si{\kelvin}$} & {$A_6 \:/\: \si{\kelvin}$}  & {$t_5 \:/\: \si{\second}$}  & {$t_6 \:/\: \si{\second}$}\\
    \midrule
    \input{build/dyn2.tex}
    \bottomrule
  \end{tabular}
\end{table}
Hieraus ergeben sich die in \ref{tab:6} angeführten Werte für die Wärmeleitfähigkeit von Aluminium.
\begin{table}
  \centering
  \caption{Einzelne Ergebnisse der Wärmeleitfähigkeit für Aluminium.}
  \label{tab:6}
  \sisetup{table-format=1.2}
  \begin{tabular}{c}
    \toprule
    {$\kappa_\text{Aluminium} \:/\: \si{\watt\per\metre\per\kelvin}$}\\
    \midrule
    \input{build/kappa_2_tab.tex}
    \bottomrule
  \end{tabular}
\end{table}
Gemittelt ergibt sich eine Wärmeleitfähigkeit von
\begin{align*}
  \kappa_\text{Aluminium} &= \input{build/kappa_2.tex}. \\
\end{align*}

\subsubsection{Bestimmung der Wärmeleitfähigkeit für Edelstahl}
Als letztes soll die Wärmeleitfähigkeit für Edelstahl bestimmt werden, wobei in diesem Fall eine veränderte Periodendauer der periodischen Erwärmung verwendet wird.
Der Graphen für die Temperaturen an den Messstellen zeigt Abbildung \ref{fig:7}
\begin{figure}[H]
  \centering
  \includegraphics[height=13cm]{scan-1.jpg}
  \caption{Zeitlicher Verlauf der Temperaturen für $T_7$ und $T_8$ bei periodischer Erwärmung nach der Angström-Methode.}
  \label{fig:7}
\end{figure}
Hier werden die Amplituden zu den in Tabelle \ref{tab:7} angegebenen Messwerten abgelesen
\begin{table}
  \centering
  \caption{Maxima der Temperaturen $T_7$ und $T_8$.}
  \label{tab:7}
  \sisetup{table-format=1.2}
  \begin{tabular}{c c c c}
    \toprule
    {$A_7 \:/\: \si{\kelvin}$} & {$A_8 \:/\: \si{\kelvin}$}  & {$t_7 \:/\: \si{\second}$}  & {$t_8 \:/\: \si{\second}$}\\
    \midrule
    \input{build/dyn3.tex}
    \bottomrule
  \end{tabular}
\end{table}
Hieraus ergeben sich die in Tabelle \ref{tab:8} angegebenen Werte für die Wärmeleitfähigkeit.
\begin{table}
  \centering
  \caption{Einzelne Ergebnisse der Wärmeleitfähigkeit für Edelstahl.}
  \label{tab:8}
  \sisetup{table-format=1.2}
  \begin{tabular}{c}
    \toprule
    {$\kappa_\text{Edelstahl} \:/\: \si{\watt\per\metre\per\kelvin}$}\\
    \midrule
    \input{build/kappa_3_tab.tex}
    \bottomrule
  \end{tabular}
\end{table}
Somit kann auf einen gemittelten Wert von
\begin{align*}
  \kappa_\text{Edelstahl} &= \input{build/kappa_3.tex} \\
\end{align*}
geschlossen werden.
%\begin{table}
%  \centering
%  \caption{Beispieltabelle}
%  \label{tab:tabelle_beispiel}
%  \sisetup{table-format=1.2}
%  \begin{tabular}{c c}
%    \toprule
%    {$a [\si{\second}]$} & {$b [\si{\kelvin}]$}\\
%    \midrule
%    1.0000  & 11.00 \\
2.0000  & 12.00 \\
3.0000  & 13.00 \\
4.0000  & 14.00 \\
5.0000  & 15.00 \\
6.0000  & 16.00 \\
7.0000  & 17.00 \\
8.0000  & 18.00 \\
9.0000  & 19.00 \\
10.0000 & 20.00 \\

%    \bottomrule
%  \end{tabular}
%\end{table}
%
%Es ergibt sich
%\begin{align}
%  a &= (0 \pm 0) ~ \si{\joule\per\kelvin\per\gram}
 \\
%\end{align}

%\section{Diskussion}
\label{sec:Diskussion}



Der aus Plot \ref{plot:6} bestimmte Quotient aus der Planckschen Konstante und der Elementarladung lautet
\begin{align*}
  \left(\frac{h}{e_0}\right)_{\text{gem}} = \input{build/a_6.tex}.
\end{align*}
Verglichen mit dem Literaturwert \cite{Konstanten},
\begin{align*}
  \left(\frac{h}{e_0}\right)_{\text{lit}} = \input{build/a_lit.tex},
\end{align*}
ergibt sich eine prozentuale Abweichung von
\begin{align*}
  \increment{\left(\frac{h}{e_0}\right)} = \input{build/abw_he.tex}.
\end{align*}
Die aus jenem Plot berechnete Austrittsarbeit beträgt
\begin{align*}
  A = \input{build/ak.tex}.
\end{align*}
Es fallen jedoch zwei Sonderheiten auf.
Zunächst sieht es so aus, als würden die Punkte des Plots eher einen quadratisch ansteigenden Trend verfolgen, welcher die Theorie wiederlegen wiederlegen würde.
Die berechnete Steigung bestätigt die Theorie wiederrum, was darauf schließen lässt, dass die bestimmten Punkte durch eine unglückliche Messdatenaufnahme vom wahren Wert abweichen.
Zum Beispiel könnte dies an der geringen Intensität und oder Fokussierung der Spektrallinien liegen. Exemplarisch lässt sich hier die blaugrüne Linie herauspicken, da sie durch eine sehr schwache Intensität gekennzeichnet ist.
Zum anderen war es schwierig die ultraviolette Linie aufzunehmen, da diese nur auf dem floureszierenden Schirm zu sehen ist.\\
Insgesamt lässt sich die Theorie zum Photoeffekt bestätigen, jedoch laden gewisse Messungenauigkeiten eher zu einem sogenannten vorsichtigen Genuss ein.

%
%\printbibliography

\section{1a)}
Die gegebene Formel beschreibt die Masse eines gesamten Atomkernes in Abhängigkeit der Massenzahl und der Ordnungzahl (Anzahl Protonen). Diese setzt sich aus der Masse der einzelnen, ungebunden Nukleonen abzüglich der in Masse umgerechneten Bindungsenergien zusammen. Wegen dem Massendefekt erhält man die fehlende Masse beim Zusammensetzen des Kernes als Bindungsenergie.\\
Annahme: Man nimmt das Tröpfenmodell des Atomkernes an, d.h. man hat Protonen und Neutronen als Kernbausteine, die durch kurzreichweitige Kernkraft zusammengehaten weden. Die Kernkraft wirkt von jedem Nukleon nur auf die benachbarten Nukleonen. Man hat also Parallelen zu einer inkompressiblen Flüssigkeit.\\
\begin{itemize}
  \item $M_1$: Die Masse der einzelnen Protonen
  \item $M_2$: Die Masse der einzelnen Neutronen.
  \item $M_3$: Die Wechselwirkung findet nur zum jeweils nächsten Nachbarn statt, deshalb steigt die Bindungsenergie linear mit der Anzahl der Nukleonen an.
  \item $M_4$: Die Nukleonen an der Oberfläche besitzen weniger Nachbarn und tragen somit weniger zur Bindungsenergie bei. Empirisch ist der mittlere Kernradius durch den Zusammenhang $R_{\frac{1}{2}} = \sqrt[3]{A}$ bekannt, die Oberfläche ist $\sim R^2$. Somit folgt $M_4 \sim A^{\frac{2}{3}}$.
  \item $M_5$: Protonen stoßen sich aufgrund ihrer Ladung ab, wodurch die Bindungsenergie im Kern verringert wird. Dabei ist die Coloumbabstoßung $\sim Z^2$ und sie fällt mit $\sim \frac{1}{R}$ ab, mit dem mittleren Kernradius ergibt sich also $M_5 \sim \frac{1}{\sqrt[3]{A}}$.
  \item $M_6$: Asymmetrieterm: Ein Ungleichgewicht der Protonen und Neutronen mindert die Stabilität und somit die Bindungsenergie. Im Allgemeinen soll $M_6 \sim (Z-N)$ gelten. Damit die Vorzeichenunabhängigkeit gegeben ist, wird der Wert zusätzlich quadriert und durch $A$ geteilt.
  \item $M_7$: Es gibt eine (in diesem Modell empirisch beründete) besondere Stabilität von gg-Kernen sowie eine Instabilität von uu-Kernen. Deshalb nimmt die Bindungsenergie ab (für uu-Kerne), bzw. zu (für gg-Kerne). Empirisch ergibt sich auch die Proportinalität $\sim \frac{1}{\sqrt{A}}$.
\end{itemize}

\section{1b)}
\begin{figure}
  \centering
  \label{fig:1}
  \includegraphics[height=6.5cm]{1b_2.pdf}
  \caption{Exemplarisch dargestellte Abhängigkeit der Masse von der Ordnungszahl für ein Nuklid mit ungerader Massenzahl.}
\end{figure}
In Abbildung 1 sind Isobare mit $A = 109$ angegeben. Eine hohe Bindungsenergie, und somit eine hohe Stabilität, äußert sich hier in einer geringen Masse des Kerns, da sich die Bindungsenergie jeweils aus der Differenz der berechneten Kernmasse und der (hier konstanten) Masse der einzelnen Nukleonen ergibt. Der Zustand ist hier energetisch am günstigsten.
Man sieht, dass es ein eindeutiges Minimum bei $Z=47$ ergibt. Dies ist auch darin bedründet, dass es sich hier wegen dem ungeraden $A$ weder um gg- noch um uu-Kerne handeln kann, so dass der Graph durch eine einzelne Parabel mit einem Minimum beschrieben wird. Dementsprechend müsste das Silberisotop der einzige stabile Kern sein, was durch ein Blick auf die Nuklidkarte bestätigt wird.


\section{1c)}

\begin{figure}
  \centering
  \label{fig:2}
  \includegraphics[height=7cm]{1c.pdf}
  \caption{Abhängigkeit der Masse von der Ordnunszahl für A = 110.}
\end{figure}
Die Abhängigkeit $M(Z)$ ist für $A=110$ in Abbildung 2 dargstellt. Man erkennt die beiden gg- und uu-Kerne, welche als Parablen dargestellt werden.
Für $Z=46$ ergibt sich also ein Minimum im Vergleich zu den benachbarten Kernen. Ein $\beta^{-}$-Zerfall (nach rechts) oder ein $\beta^{+}$-Zerfall (nach links) wären beide also ungünstig, weil Kerne mit geringerer Bindungsenergie entstehen würden. Dementprechend handelt es sich bei Palladium mit der Massenzahl $A=110$ also um ein stabiles Nuklid, was ein Blick auf die Nuklidkarte bestätigt.

\section{1d)}
\begin{figure}
  \centering
  \label{fig:3}
  \includegraphics[height=7cm]{1d.pdf}
  \caption{Abhängigkeit der Masse von der Ordnungszahl für A=8.}
\end{figure}
Man sieht in Abbildung 3, dass stabile Kerne für $Z=4$ vorausgesagt werden. Die Begründung würde analog zu den vorherigen Aufgabenstellungen erfolgen. Ein Blick auf die Nuklidkarte offenbart, dass der Kern jedoch nicht stabil ist!
Dies liegt darin begründet, dass der Kern in zwei Heliumatome mit $Z=N=2$ zerfallen kann. Hierbei handelt es sich um einen doppelten magischen Kern, bei dem man eine besondere Stabilität zu erwarten hat, die die Voraussagen der Bethe-Weizsäcker-Formel noch mal stark übertrifft.
Deshalb ist der Kern gegen der eigentlichen Erwartung instabil.




\end{document}
