\section{Diskussion}
\label{sec:Diskussion}


% G hat abweichung von 25%, wegen wackelig oder zu große Auslenkung, aber wenn man die Werte der Zeiten betrachtet, scheinen sie nur kleine statistische Schwankungen zu haben, also kann der Fehler nur an dem Messverfahren generell liegen....
% Dementsprechend sind alle anderen Konstanten kacke
% Am B-Feld der Erde scheitert es an den Zeiten, die noch viel zu groß sind, wenn man D_echt einsetzt kommt nämlich was negatives heraus
% Fazit: Voll der Bullshit :/



Der ermittelte Schubmodul beträgt $G = \input{build/schubmodul.tex}$.
Der echte Referenzwert beträgt jedoch $G_{\text{echt}} = \input{build/schubmodul_echt.tex}$.
Es zeigt sich demnach eine Abweichung von $\increment G = \input{build/g_abweichung.tex}$.
Das Problem ist, dass die gemessenen Periodendauern nur kleine statistische Abweichungen zeigen.
Dadurch müsste der Schubmodul passend bestimmt sein.
Daraus lässt sich schließen, dass die Methode, mit der der Modul bestimmt wurde nicht passend ist.
Bei der Bestimmung der anderen Elastizitätskonstanten geht nun leider der fehlerhaft bestimmte Schubmodul ein.
Es folgt bei der Querkontraktionszahl eine Abweichung zur echten $\mu_{\text{echt}} = \input{build/querkontraktionszahl_echt.tex}$ von $\increment \mu = \input{build/mu_abweichung.tex}$.
Diese hohe Abweichung erklärt den negativen Kompressionsmodul, welches aus der vorher ermittelten Querkontraktionszahl bestimmt wird.
Das bestimmte Magnetfeld der Erde, $B_{\text{Erde}}=\input{build/magnetfeld_erde.tex}$, ist um ungefähr den Faktor Zehn zu groß.
Dies liegt jedoch nicht nur an dem fehlerhaften Schubmodul, welches die Richtgröße beeinflusst, sondern zusätzlich an den gemessenen Periodendauern.
Diese weisen nämlich keine statistischen Schwankungen auf, aber einen deutlichen Trend nach unten.
Der Torsionsfaden ist zu weit ausgelenkt worden, sodass die Zeiten von der Amplitude abhängig sind.
Dies ist ein systematischer Fehler, der hätte vermieden werden sollen.
Zuletzt steht die Frage offen, ob dieser Versuchsaufbau wirklich dazu geeignet ist, die Elastizitätskonstanten zu bestimmen.
