\subsection{Durchführung}
\label{sec:durchführung}

Bei jeder Untersuchung wird die Spannungszufuhr der Röntgenröhre auf $U_{\text{B}} = \SI{35}{\kilo\volt}$, sowie deren Stromstärke auf $I = \SI{1}{\milli\ampere}$ gestellt.

\subsubsection{Überprüfung der Bragg Bedingung}
Um die Bragg Bedingung zu prüfen, wird zunächst der LiF-Kristall auf einen festen Winkel $\SI{14}{\degree}$ relativ zur Strahllinie gestellt.
Das Zählrohr fährt den Winkelbereich von von $\SI{26}{\degree}$ bis $\SI{30}{\degree}$ in $\si{0,1}$-ser Schritten mit einer Integrationszeit von $\SI{20}{\second}$ ab.

\subsubsection{Emissionsspektrum der Kupfer-Röntgen-Röhre}
Zum Messen des Emissionsspektrums und auch für folgende Messungen wird die Messmethode auf den 2:1 Modus umgestellt.
Die Drehung des Kristalls soll von $\SI{4}{\degree}$ bis $\SI{26}{\degree}$ stattfinden, diesmal jedoch in $\si{0,2}$-ser Schritten  mit einer Integrationszeit von $\SI{5}{\second}$.

\subsubsection{Absorptionsspektren}
Bevor Absorptionsspektren aufgenommen werden können, wird die Öffnung des Zählrohrs mit dem entsprechenden Absorber versehen.
Die Messung erfolgt erneut in $\si{0,1}$-ser Schritten mit einer Integrationszeit von $\SI{20}{\second}$.
Dabei wird versucht, den Teil des Spektrums aufzunehmen, in der die K-Kante zu sehen ist.
Dies wird für Germanium, Strontium und Zirkonium getan.
Bei der letzten Messung mit Wismut wird ein Spektrum aufgenommen, welches die ersten drei L-Kanten beinhaltet.
