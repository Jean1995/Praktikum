\section{Diskussion}
\label{sec:Diskussion}
Bei der Analyse des Emissionsspektrums der Röntgenröhre fällt auf, dass sich Abweichungen von jeweils
\begin{align*}
  \increment E_{K_\alpha} &= \SI{1.2}{\percent}
 \\
  \increment E_{K_\beta} &= \SI{1.7}{\percent}

\end{align*}
ergeben haben, wobei es sich jeweils um Abweichungen der Messenergien nach oben handelt.
Außerdem weicht der Wert der maximalen Energie, bei einer angenommenen Beschleunigungsspannung von $\SI{35}{\kilo\volt}$ um
\begin{align*}
  \increment E_{\text{max}} &= \SI{7.3}{\percent}

\end{align*}
ab.
Diese konstanten Messfehler in eine Richtung, welche sich auch bei der Analyse der Absorptionsspektren ergeben, lassen einen systematischen Messfehler wahrscheinlich erscheinen.\\
Bei der Bestimmung der Abschirmkonstanten von Kupfer ergab sich eine relative Abweichung von
\begin{align*}
  \increment \sigma_{\text{K}_{\alpha}} &= \SI{3.2}{\percent}
, \\
  \increment \sigma_{\text{K}_{\beta}} &= \SI{1.8}{\percent}
.
\end{align*}
Hier liegen die bestimmten Werte für $\sigma$ unterhalb der Literaturwerte, was wiederum auf einen systematischen Fehler hinweist.\\
Ebenso fehlerbehaftet ist die bestimmte Rydbergkonstante, welche vom Literaturwert \cite{Konstanten} mit
\begin{align*}
  \increment R_\infty &= \SI{33.3}{\percent}
, 
\end{align*}
abweicht.
Zu den bereits genannten Einflüssen durch systematische Messfehler trägt hier auch die geringe Anzahl an Messwerten für die lineare Ausgleichsrechnung zum Fehler bei.
