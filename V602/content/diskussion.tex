\section{Diskussion}
\label{sec:Diskussion}
Bei Mehrelektronenatomen wird die Kernladung durch die Elektronen
abgeschirmt, sodass die Elektronen in den äußeren Schalen eine geringere
Bindungsenergie besitzen als in den inneren Schalen und somit leichter
herausgeschlagen werden können. Durch Bestimmung der K-Kanten kann
mithilfe von Formel \eqref{eq:sigma_k} die Abschirmzahl
$\sigma_\mathrm{K}$ bestimmt werden. Dabei bezeichnet $\alpha$ die
Feinstrukturkonstante, $Z$ die Kernladungszahl $E_\text{K}$ die Energie
der K-Kante und $R_\infty$ die Rydberg-Konstante.
%
\begin{equation}
  \sigma_\text{K} = Z - \sqrt{\frac{E_\text{K}}{h c R_\infty}
    - \frac{\alpha^2 Z^4}{4}}
\label{eq:sigma_k}
\end{equation}
% 
