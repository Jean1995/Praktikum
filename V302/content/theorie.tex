\section{Theorie}
\subsection{Grundlagen der Brückenschaltungen}
\label{sec:Theorie}
Bei einer Brückenschaltung können im Allgemeinen unbekannte Größen bei Kenntnis ausreichend vieler Größen, insbesondere der Speisespannung $U_s$ sowie der Brückenspannung $U_b$, mit Hilfe der Kirchhoffschen Regeln bestimmt werden.\\
Die erste Kirchhoffsche Regel ist dabei die Knotenregel \eqref{eqn:knotenregel}.
Diese folgt direkt aus der Ladungserhaltung und besagt, dass in jedem Knotenpunkt einer Schaltung die Summe alle eingehenden und ausgehenden Ströme verschwinden muss:

\begin{equation}
  \sum_{k=1}^n I_k = 0.
  \label{eqn:knotenregel}
\end{equation}
Beispielsweise gilt für Abbilung \ref{fig:knotenregel}
\begin{equation}
I_1 = I_2 + I_3
\end{equation}
\begin{figure}[H]
  \centering
  \begin{tikzpicture}
    \draw[->] (0, 0) -- (1.8, 0) node[midway, above] {$I_1$};
    \draw[->] (2, 0.1) -- (4, 1) node[midway, above] {$I_2$};
    \draw[->] (2, 0) -- (4, 0) node[midway, below] {$I_3$};
    %\draw[IPfeil=0em]([yshift=1.0em]0,0) -- node [above]{I$\mathsf{_R}$}([yshift=1.0em]0.5,0);
  \end{tikzpicture}
  \caption{Maschenregel.}
  \label{fig:knotenregel}
\end{figure}
Die zweite Kirchhoffsche Regel ist die Maschenregel \eqref{eqn:maschenregel}.
Sie folgt aus dem Induktionsgesetz im Vakuum
\begin{equation}
\oint \vec{E} \cdot \vec{\symup{d}s} = 0
\end{equation}
und besagt, dass in jeder geschlossenen Masche der Schaltung die Summe der Spannungen null ergeben muss:
\begin{equation}
  \sum_{i=1}^n U_1 = 0.
  \label{eqn:maschenregel}
\end{equation}
Beispielsweise gilt für Abbildung \ref{fig:maschenregel}
\begin{equation}
U_s = I \cdot R_3 + I \cdot R_2 + I \cdot R_1
\end{equation}
\begin{figure}[H]
  \centering
  \begin{tikzpicture}[circuit ee IEC, font=\sffamily\footnotesize]
      \draw (0,0) to [resistor={info={$R_1$},info'={$\mathsf{_{}}$}}] (4,0);
      \draw (4,0) to [resistor={info={$R_2$},info'={$\mathsf{_{}}$}}] (4,2);
      \draw (4,2) to [resistor={info={$R_3$},info'={$\mathsf{_{}}$}}] (0,2);
      \draw (0,2) to [battery={info'={$U_s$}}] (0,0);
      %\node at ([xshift=-1em]battery.north) {+};

      \draw[IPfeil=0em]([yshift=1.0em]1.7,2.1) -- node [above]{$I$}([yshift=1.0em]2.3,2.1);
    \end{tikzpicture}
    \caption{Maschenregel.}
    \label{fig:maschenregel}
\end{figure}

Um nun eine unbekannte Größe zu messen, sieht eine allgemeine Brückenschaltung wie in Abbildung \ref{fig:bruecke} beschrieben aus.

%Volt- und Amperemeter festlegen:
\tikzset{circuit declare symbol = Us}
\tikzset{set Us graphic ={draw,generic circle IEC, minimum size=5mm,info=center:$U_s$}}

\tikzset{circuit declare symbol = Ub}
\tikzset{set Ub graphic ={draw,generic circle IEC, minimum size=5mm,info=center:$U_b$}}

\begin{figure}[H]
  \centering
  \begin{tikzpicture}[circuit ee IEC, font=\sffamily\footnotesize]

      \draw(1,2) to (0,2);
      \draw(0,2) to (0,5);


      \draw (1,0) to [resistor={info={$Z_3$},info'={$\mathsf{_{}}$}}] (4,0);
            \draw[IPfeil=0em]([yshift=1.0em]1.1,-0.2) -- node [above]{$I_3$}([yshift=1.0em]1.6,-0.2);
      \draw[fill=black] (4,0) circle (1.5pt);
      \draw (4,0) to [resistor={info={$Z_4$},info'={$\mathsf{_{}}$}}] (7,0);
      \draw[IPfeil=0em]([yshift=1.0em]6.1,-0.2) -- node [above]{$I_4$}([yshift=1.0em]6.6,-0.2);
      \draw (7,0) to (7,2);

      \draw(1,0) to (1,2);
      \draw[fill=black] (1,2) circle (1.5pt);
      \draw(1,2) to (1,4);
      \draw (1,4) to [resistor={info={$Z_1$},info'={$\mathsf{_{}}$}}] (4,4);
      \draw[IPfeil=0em]([yshift=1.0em]1.1,3.8) -- node [above]{$I_1$}([yshift=1.0em]1.6,3.8);
      \draw[fill=black] (4,4) circle (1.5pt);
      \draw (4,4) to [resistor={info={$Z_2$},info'={$\mathsf{_{}}$}}] (7,4);
      \draw[IPfeil=0em]([yshift=1.0em]6.1,3.8) -- node [above]{$I_2$}([yshift=1.0em]6.6,3.8);
      \draw (7,4) to (7,2);
      \draw[fill=black] (7,2) circle (1.5pt);

      \draw(7,2) to (8,2);
      \draw(8,2) to (8,5);

      \draw (4,0) to  [Ub={info'={}}](4,4);
      \draw (8,5) to  [Us={info'={}}](0,5);


      %\draw (4,2) to [resistor={info={$R_3$},info'={$\mathsf{_{}}$}}] (0,2);
      %\draw (0,2) to [battery={info'={$U_s$}}] (0,0);
      %\node at ([xshift=-1em]battery.north) {+};
      %\draw[IPfeil=0em]([yshift=1.0em]1.7,2.1) -- node [above]{$I$}([yshift=1.0em]2.3,2.1);
    \end{tikzpicture}
    \caption{Allgemeine Brückenschaltung.}
    \label{fig:bruecke}
\end{figure}
Damit die folgenden Formeln, auch wenn für $U_s$ ein Wechselstrom angelegt wird, gelten, wird die Impedanz
\begin{equation}
  Z = R + i X
\end{equation}
genutzt.
Hierbei steht $R$ für den Wirkwiderstand und $X$ für den Blindwiderstand. \\
Nach der Knotenregel gelten für die Ströme, wenn der Innenwiderstand des Voltmeters von $U_b$ so gewählt wird dass er keinen Storm durchlässt, die beiden Zusammenhänge
\begin{gather}
  I_1 = I_2 \\
  I_3 = I_4
\end{gather}
sowie nach der Maschenregel
\begin{align}
  U_b - Z_3 I_3 + Z_1 I_1 &= 0 \\
  U_b + Z_4 I_4 - Z_2 I_2 &= 0
\end{align}
Werden beide Ausdrücke kombiniert ergibt sich für das Verhältnis von $U_b$ zu $U_s$ der Ausdruck
\begin{equation}
  U_b = \frac{Z_2 Z_3 - Z_1 Z_4}{(Z_3 + Z_4)(Z_1 + Z_2)}U_s.
\end{equation}
Wenn der Zähler dieses Ausdruckes nun 0 wird, also das Verhältnis
\begin{equation}
Z_2 Z_3 - Z_1 Z_4 = 0 \iff Z_2 = \frac{Z_1 Z_4}{Z_3} \label{eqn:1}
\end{equation}
gilt, so wird die Brückenspannung bei anliegender Speisespannung 0.\\
Dieses Nullmethode genannte Verfahren kann genutzt werden, um eine unbekannte Impedanz $Z_2$ bei ansonsten bekannten Impedanzen zu berechnen.
Handelt es sich um reele Impedanzen, d.h. die Blindwiderstände sind 0, kann dieses Verhältnis direkt genutzt werden.
Dabei wird einer der bekannten Widerstände variiert bis $U_b=0$ ist.\\
Handelt es sich um komplexe Widerstände, müssen sowohl die Realteile als auch die Imaginärteile 0 ergeben.
Das bedeutet, dass die beiden Gleichungen
\begin{equation}
  R_1 R_4 - X_1 X_4 = R_2 R_3 - X_2 X_3
  \label{eqn:bed1}
\end{equation}
\begin{equation}
  R_1 X_4 + R_4 X_1 = R_2 X_3 + R_3 X_2
  \label{eqn:bed2}
\end{equation}

simultan erfüllt sein müssen.
Beim Nullabgleich müssen dementsprechend zwei bekannte Werte variiert werden, so dass die Brückenspannung 0 wird.
%\cite{sample}
\subsection{Impedanzen bekannter elektrischer Bauelemente}
Die Impedanzen eines ohmschen Widerstandes $R$, einer Induktivität $L$ sowie einer Kapazität $C$ sind gegeben durch
\begin{equation}
  Z_R = R,
\end{equation}
\begin{equation}
  Z_L = i \omega L
\end{equation}
sowie
\begin{equation}
  Z_C = \frac{-i}{\omega C}.
\end{equation}\cite{sample}
