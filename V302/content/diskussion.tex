\section{Diskussion}
\label{sec:Diskussion}
Bei der Induktivitätsmessung geht man in der Theorie davon aus, dass mithilfe eines verlustarmen Kondensators anstelle einer Spule eine genauere Messung durchgeführt werden kann.
Unsere Messungen der unbekannten Widerstände und Induktivitäten weisen bei den beiden unterschiedlichen Messmethoden eine gewisse Abweichung auf.
Die Induktivität wurde bei der Induktivitätsbrücke kleiner bestimmt als bei der Maxwellbrücke, während der Widerstand größer ist.
Der größere Widerstand und die geringere Kapazität lassen sich auf einen generell höheren Verlust einer Spule im Vergleich zu einem Kondensator zurückführen. \\
Unsere Messung bei der Maxwellbrücke ergab einen höheren Fehler, was sich möglicherweise darauf zurückführen lässt, dass wir hier mehrere Messungen durchgeführt haben.
Gleichzeitig haben wir, mangels fehlender Vergleichsspulen, lediglich zwei Messungen bei der Induktivitätsmessbrücke durchgeführt, welche scheinbar zufällig nah beieinander gelegen haben.\\
Die Messungen mittels Wheatstonebrücke sowie Kapazitätsmessbrücke verliefen ohne signifikante Auffälligkeiten. \\
Bei der frequenzabhängigen Messung gilt es die gemessene Sperrfrequenz $v_0$ mit der errechneten zu vergleichen.
Während der errechnete Wert bei $\SI{1149.6}{\hertz}$ liegt, ergab sich bei unserer Messung ein Wert von $\SI{1125}{\hertz}$.
Diese Abweichung lässt sich dadurch erklären, dass wir bei der Frequenzeinstellung lediglich in $\SI{25}{\hertz}$ Schritten genau sein konnte.\\
Auffällig ist ebenfalls das Abweichung unserer Messwerte von der Theoriekurve, vor allem bei höheren Frequenzen als der Sperrfrequenz.
Dies ist ein Phänomen, welches wir nicht erklären können.
Entscheidend ist jedoch, dass wir eine deutliche Bandsperre für die Sperrfrequenz erkennen können.
Weitere Ungenauigkeiten folgen aus schaltungsinternen Fehlerquellen, beispielsweise Verlusten an Kontakten, reale Widerstände von Kabeln, etc.
