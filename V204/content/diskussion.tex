\section{Diskussion}
\label{sec:Diskussion}
Die qualitativen Ergebnisse der Wärmeleitfähigkeit stimmen bei der statischen Methode mit den Erwartungen überein.
Bei der Angström Methode ergeben sich jedoch starke Abweichungen in den Ergebnissen der einzelnen Messungen, welche sich in einem dementsprechend hohen Fehler spiegeln.
Die ermittelten Messwerte betragen

\begin{align*}
  \kappa_\text{Messing} = \input{build/kappa_1.tex} \\
  \kappa_\text{Aluminium} = \input{build/kappa_2.tex},\\
  \kappa_\text{Edelstahl} = \input{build/kappa_3.tex}.
\end{align*}
Im Vergleich hierzu können Literaturwerte von
\begin{align*}
  \kappa_\text{Messing, lit} = \SI{120}{\watt\per\metre\per\kelvin}, \\
  \kappa_\text{Aluminium, lit} = \SI{236}{\watt\per\metre\per\kelvin}, \\
  \kappa_\text{Edelstahl, lit} = \SI{15}{\watt\per\metre\per\kelvin}, \\
\end{align*}
gefunden werden.
Die Abweichungen der jeweiligen Messerte zu den Theoriewerten betragen folgendermaßen
\begin{align*}
  \increment \kappa_\text{Messing} = \input{build/fehler_1.tex} \si{\percent}\\
  \increment \kappa_\text{Aluminium} = \input{build/fehler_2.tex} \si{\percent},\\
  \increment \kappa_\text{Edelstahl} = \input{build/fehler_3.tex} \si{\percent}.
\end{align*}
Das Ablesen der Maxima und Minima stellte sich als problematisch dar:
Da die Messwerte für die dynmaischen Messunden nicht abgelesen werden konnten, mussten diese manuell anhand vom Graphen bestimmt werden.
Dies führt zu merklichen Defiziten bei der Genauigkeit der Daten.
Zudem hat es sich bei der Messung für Edelstahl als schwierig erwiesen, die Amplituden für die Temperaturen von $T_8$ bei der Edelstahlmessung zu bestimmen, da es sich annähernd um stationäre Stellen handelt, so dass das Bestimmen einer Amplitude nicht genau möglich war.
