\section{Zielsetzung}
Im Folgenden Experiment wird das Phänomen der Wärmeleitung anhand von mehreren Metallen betrachtet.
Dabei soll der zeitliche Temperaturverlauf sowie das Verhalten unter einer periodischen Anregung betrachtet werden.
Zudem wird die Wärmeleitfähigkeit mithilfe der Angström-Methode für Aluminium, Messing sowie Edelstahl bestimmt.

\section{Theorie}
\subsection{Herleitung der eindimensionalen Wärmeleitungsgleichung}
Existiert an einem Material, beispielsweise einem Metall, eine Temperaturdifferenz zwischen zwei Orten, so findet erfahrungsgemäß ein Temperaturausgleich statt.
Dieser findet in einem Metall in Form der Wärmeleitung statt, bei der vornehmlich die frei beweglichen Elektronen die Wärme transportieren.
Dementsprechend besitzen Metalle bekanntermaßen eine bessere Wärmeleitfähigkeit $\kappa$ als Nicht-Metalle.\\
Für einen Stab, an dem eine Temperaturdifferenz anliegt, wird die in einer Zeit $\symup{d}t$ durch einen Querschnitt $A$ fließende Wärmemenge $\symup{d}Q$ durch das Verhältnis
\begin{equation}
  \symup{d}Q = - \kappa A \frac{\partial T}{\partial x} \symup{d}t
  \label{eqn:w}
\end{equation}
beschrieben, wobei $L$ die Länge des betrachteten Stabs beschreibt.
Hieraus lässt sich die Wärmestromdichte $j_w$ als
\begin{equation}
  j_w = - \kappa \frac{\partial T}{\partial x}
\end{equation}
definieren.\\
Es wird außerdem die Kontinuitätsgleichung
\begin{equation}
  \frac{\partial \rho_q}{\partial t} + \nabla \vec{j_w} = 0,
\end{equation}
betrachtet, wobei $\rho_q$ die spezifische Wärmemenge $\frac{\symup{d}Q}{\symup{d}V}$ pro Volumen bezeichnet.
Die Definition der spezifischen Wärmekapazität
\begin{equation}
  c = \frac{\symup{d}Q}{m \symup{d}T}
\end{equation}
ergibt, zusammen mit der eindimensionalen Version der Kontinuitätsgleichung, die eindimensionale Wärmeleitungsgleichung
\begin{equation}
  \frac{\partial T}{\partial t} = \frac{\kappa}{\rho c} \frac{\partial^2 T}{\partial x^2}.
  \label{eqn:dgl}
\end{equation}
Hier beschreibt $\rho$ nun die Dichte des verwendeten Materials, so dass der Vorfaktor als Materialkonstante
\begin{equation}
  \sigma_T = \frac{\kappa}{\rho c}
\end{equation}
zusammengefasst werden kann.
Dieser Wert wird auch als Temperaturleitfähigkeit bezeichnet und gibt Aufschluss darüber, mit welcher Geschwindigkeit ein Temperaturausgleich stattfindet.\\
\subsection{Anregen einer Temperaturwelle durch periodische Wärmeanregung}
Das periodische Anregen des Körpers mit Wärme führt zu einer sich im Stab ausbreitenden Temperaturwelle
\begin{equation}
  T(x,t) = T_{\text{max}} \exp{\Bigl( -\sqrt{\frac{\omega \rho c}{2 \kappa} } x \Bigr)} \cos{\Bigl( \omega t - \sqrt{\frac{\omega \rho c}{2 \kappa}} x \Bigr)},
\end{equation}
wobei die Kreisfrequenz
\begin{equation}
  \omega = \frac{2 \pi}{T}
\end{equation}
von der gewählten Periodendauer der Anregung abhängt.
Die entstehende Welle weist eine Phasengeschwindigkeit von
\begin{equation}
  v = \frac{\omega}{k} = \sqrt{\frac{2 \kappa \omega}{\rho c}}
\end{equation}
auf, wobei die Wellenzahl $k$ direkt der oben genannten Wellengleichung entnommen werden kann.
Aus der Betrachtung der Dämpfung, also dem Amplitudenverhältnis $A_1$ zu $A_2$ an zwei verschiedenen Orten mit Abstand $\increment{x}$, leitet sich die Wärmeleitfähigkeit
\begin{equation}
  \kappa = \frac{\rho c (\increment{x})^2}{2 \increment{t} \ln{ \frac{A_1}{A_2} }}
  \label{eqn:wlf}
\end{equation}
her.
Hierbei beschreibt $\increment{t}$ die Zeit, in der die Welle den Abstand $\increment{x}$ zurücklegt.\\
\subsection{Funktionsweise des Peltierelementes}
Die periodische Anregung des Stabes wird mithilfe eines Peltierelementes durchgeführt.
Dessen Funktionsweise basiert auf dem Peltier-Effekt, für den zwei Arten von Halbleitern mit unterschiedlichen Energieniveaus benötigt werden.
Durch das alternierende Anordnen dieser Elemente und des Anlegen einer Spannung fließt ein Strom.
Um in das jeweils nächste Leitungsband eintreten zu können, muss aufgrund der Energiedifferenzen der Leitungsbänder ebendiese Energie aufgenommen oder abgegeben werden.
Dies führt jeweils zu Abkühlung oder Erwärmung der Umgebung und kann somit zur Wärmung oder Kühlung genutzt werden.
Die Umkehrung der Richtung des Wärmetransportes kann einfach durch Umpolung der Spannung erreicht werden.

\label{sec:Theorie}

%\cite{sample}
