\subsection{Durchführung}
\label{sec:durchführung}

\subsubsection{Verifizierung des theoretischen Zusammenhanges und Bestimmung der Kenngrößen}
Um den theoretisch erwarteten linearen Zusammenhang des Photoeffekts bestätigen zu können, wird die kinetische Energie der Elektronen in Abhängigkeit von der Wellenlänge untersucht.
Dazu wird für fünf verschiedene Wellenlängen die Grenzspannung $U_g$ ermittelt.\\
Bei dessen Bestimmung mit der Gegenfeldmethode ist zu beachten, dass das Erreichen von $U_g$ nicht durch einen eindeutigen Abfall des Stroms zwischen Kathode und Anode gekennzeichnet ist.
Dies liegt daran, dass die herausgelösten Elektronen keine einheitliche kinetische Energie besitzen, sondern nach der Fermi-Dirac-Statistik eine kontinuierliche Energieverteilung.
Deshalb wird zur Bestimmung von $U_g$ bei jeder Wellenlänge der Photostrom in Abhängigkeit von der angelegten Spannung für zehn Werte gemessen.
Diese orientieren sich jeweils in der Nähe des niedrigsten messbaren Stromes.

\subsubsection{Untersuchung des Photostromes in Abhängigkeit von der Spannung}
Für eine Wellenlänge wird der Photostrom in Abhängigkeit von der Spannung betrachtet.
Dabei werden Spannungen sowohl im positiven als auch im negativen Bereich untersucht, so dass es auch zu einer Umpolung des Stroms kommt.
