\section{Diskussion}
\label{sec:Diskussion}



Der aus Plot \ref{plot:6} bestimmte Quotient aus der Planckschen Konstante und der Elementarladung lautet
\begin{align*}
  \left(\frac{h}{e_0}\right)_{\text{gem}} = \input{build/a_6.tex}.
\end{align*}
Verglichen mit dem Literaturwert \cite{Konstanten},
\begin{align*}
  \left(\frac{h}{e_0}\right)_{\text{lit}} = \input{build/a_lit.tex},
\end{align*}
ergibt sich eine prozentuale Abweichung von
\begin{align*}
  \increment{\left(\frac{h}{e_0}\right)} = \input{build/abw_he.tex}.
\end{align*}
Die aus jenem Plot berechnete Austrittsarbeit beträgt
\begin{align*}
  A = \input{build/ak.tex}.
\end{align*}
Es fallen jedoch zwei Sonderheiten auf.
Zunächst sieht es so aus, als würden die Punkte des Plots eher einen quadratisch ansteigenden Trend verfolgen, welcher die Theorie widerlegen würde.
Die berechnete Steigung bestätigt die Theorie wiederum, was darauf schließen lässt, dass die bestimmten Punkte durch eine unglückliche Messdatenaufnahme vom wahren Wert abweichen.
Zum Beispiel könnte dies an der geringen Intensität oder der Fokussierung der Spektrallinien liegen. Exemplarisch lässt sich hier die blaugrüne Linie nennen, welche durch eine sehr schwache Intensität gekennzeichnet ist.
Zum anderen war es schwierig, die ultraviolette Linie aufzunehmen, da diese nur auf dem fluoreszierenden Schirm zu sehen ist.\\
Insgesamt lässt sich die Theorie zum Photoeffekt bestätigen, jedoch laden gewisse Messungenauigkeiten eher zu einem sogenannten vorsichtigen Genuss ein.
