\subsection{Durchführung}
\label{sec:durchführung}

Bevor die Messung beginnen kann, muss das Michelson-Interferometer zunächst kalibriert werden.
Dazu wird die nicht am Motor angeschlossene Platte solange eingestellt, bis ein gemeinsamer Punkt des vom Spiegel $S_1$ sowie vom Spiegel $S_2$ kommendes Lichtstahls ausreichender Intensität gefunden werden kann.
Dass diese Eigenschaft erfüllt ist, wird simpel durch ein nacheinander stattfindendes Zuhalten der beiden Strahlengänge überprüft.
Sobald ein solcher Punkt gefunden ist, wird der Detektor auf diesen ausgerichtet, so dass das Photoelement das Licht aufnimmt.
Mithilfe einer Sammellinse wird der Lichtstrahl zudem fokussiert.
Beim allen Messungen ist zudem darauf zu achten, dass der Versuchsaufbau nicht durch Berührung des Tisches oder durch Luftzüge in Vibrationen versetzt wird, da ansonten starke Abweichungen vom Zählwerk gemessen werden die die Messung verfälschen.

\subsubsection{Bestimmung der Wellenlänge des verwendeten Lasers}
Zur Bestimmung der Wellenlänge des Lasers wird die Weglänge variiert.
Dazu wird der Synchronmotor eingestellt und die Anzahl der vom Zählwerk registrierten Impulse in einem bestimmten Wegintervall, welches an der Mikrometerschraube abgelesen werden kann, bestimmt.
Es ist dabei darauf zu achten, dass die Geschwindigkeit am Motor nicht zu schnell eingestellt wird, da das Zählwerk die Impulse ansonsten nicht mehr zuverlässig messen kann.
Diese Messung wird mehrfach für verschiedene Geschwindigkeiten und für hohe Impulszahlen wiederholt.

\subsubsection{Bestimmung des Brechungsindex}
Zur Bestimmung des Brechungsindex von Luft wird der Druck in der Messzelle variiert.
Mithilfe einer handbetriebenen Vakuumpumpe kann der Druck an einem Manometer abgelesen und verändert werden.
Zunächst wird ein Niederdruck in der Messzelle erzeugt und danach durch leichtes öffenen des Ventils ein langsamer Druckausgleich ermöglicht.
Zwischen zwei festgelegten Druckdifferenzen wird die Anzahl der gemessenen Impulse bestimmt.
Die Messung wird mehrmals wiederholt.\\
Zur Bestimmung des Brechungsindexes für ein weiteres Gas, hier $\ce{C4H8}$, wird die Messzelle zunächst bestmöglich evakuiert.
Daraufhin wird sie mit dem Gas gefüllt.
Für mehrere Messungen wird nun der oben beschriebene Vorgang wiederholt, wobei die Messzelle jeweils nicht mit Luft, sondern mit dem Gas aus der Gasflasche den Durchausgleich durchführt.
