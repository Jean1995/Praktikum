\section{Diskussion}
\label{sec:Diskussion}
Die Messergebnisse, sowie Literaturwerte \cite{dingens} \cite{kirchen} und Abweichungen sind im Folgenden nochmals dargestellt, wobei die erste Zeile für Luft und die zweite für $\ce{C4H8}$ steht.
\begin{align*}
  \lambda_{\text{Laser}} &= \input{build/lambda_laser.tex}\\
\end{align*}
\begin{table}
    \centering
    \caption{Messergebnisse.}
    \label{tab:4}
    \sisetup{parse-numbers=false}
    \begin{tabular}{
	S[table-format=1.6]
	S[table-format=1.6]
	S[table-format=1.6]
	S[table-format=2.3]
	}
	\toprule
	{$n$}		& {$\Delta n$}		& 
	{$n_{\text{lit}}$}		& {$\delta n \:/\: \si{\percent}$}		\\ 
	\midrule
    1.000355 & 0.000050 & 1.000292 & 0.006  \\
1.001484 & 0.000225 & 1.381100 & 27.486 \\

    \bottomrule
    \end{tabular}
    \end{table}
\\
%n_{\text{Luft}} &= \input{build/n_luft.tex},\\
%n_{\text{Luft,lit}} &= \input{build/n_luft_lit.tex},\\
%\Delta n_{\text{Luft}} &= \input{build/n_luft_rel.tex},\\
%n_{\ce{C4H8}} &= \input{build/n_gas.tex},\\
%n_{\ce{C4H8,lit}} &= \input{build/n_gas_lit.tex} \: \text{und}\\
%\Delta n_{\ce{C4H8}} &= \input{build/n_gas_rel.tex}.\\
Zunächst ist der unverhältnismäßig große Wellenlängenwert für das verwendete rote Laserlicht, sowie dessen hohe Unschärfe auffällig.
Die Ursache für diese Abweichungen liegt in den beiden großen, abweichenden Messwerten in Tabelle \ref{tab:1}.
Da diese im Vergleich einen systematischen Fehler darstellen, werden diese nun vernachlässigt und es ergibt sich ohne deren Berücksichtigung eine deutlich bessere Wellenlänge von
\begin{align*}
  \lambda_{\text{Laser,neu}} &= \input{build/lambda_laser_x.tex}.
\end{align*}
Die beiden abweichenden Werte, zu finden in der siebten und achten Zeile, sind vermutlich deswegen so hoch, da sie bei der höchsten Geschwindigkeit des Schiebers aufgenommen wurden.
Bei dieser hohen Geschwindigkeit beginnt die Zahnradkonstruktion zu wackeln, was in einer zu hohen Zählrate endet.\\
Durch diesen neuen Wert ergeben sich auch verbesserte Werte für die Messung des Brechungsindex für Luft
\begin{align*}
  n_{\text{Luft,neu}} &= \input{build/n_luft_neu.tex},\\
  \Delta n_{\text{Luft,neu}} &= \input{build/n_luft_rel_neu.tex}.
\end{align*}
Der ermittelte Wert für $\ce{C4H8}$ weist trotzdem eine hohe Abweichung im Vergleich zum Literaturwert auf.
Während der Messung ist auffällig gewesen, dass der Druck in dem Gasbehälter auch ohne Nutzung der Pumpe permanent dem Normaldruck entgegen strebte.
Dieser scheint dementsprechend nicht hinreichend gut abgedichtet zu sein, sodass während der Messung von außen zu viele Luftmoleküle hinein diffundierten, wodurch die Messwerte nach unten hin verfälscht wurden.
