\section{Diskussion}
\label{sec:Diskussion}
Die Messergebnisse, sowie Literaturwerte \cite{dingens} \cite{kirchen} und Abweichungen lauten
\begin{align*}
  \lambda_{\text{Laser}} &= \input{build/lambda_laser.tex},\\
  n_{\text{Luft}} &= \input{build/n_luft.tex},\\
  n_{\text{Luft,lit}} &= \input{build/n_luft_lit.tex},\\
  \Delta n_{\text{Luft}} &= \input{build/n_luft_rel.tex},\\
  n_{\ce{C4H8}} &= \input{build/n_gas.tex},\\
  n_{\ce{C4H8,lit}} &= \input{build/n_gas_lit.tex} \: \text{und}\\
  \Delta n_{\ce{C4H8}} &= \input{build/n_gas_rel.tex}.\\
\end{align*}
Zunächst ist der unverhältnismäßig große Wellenlängenwert für das verwendete rote Laserlicht, sowie dessen hohe Unschärfe auffällig.
Dies liegt in den beiden großen Messwerten in Tabelle \ref{tab:1} begründet.
Da diese im Vergleich einen systematischen Fehler darstellen, werden diese nun entfernt und es ergibt sich eine deutlich bessere Wellenlänge von
\begin{align*}
  \lambda_{\text{Laser,neu}} &= \input{build/lambda_laser_x.tex}.
\end{align*}
Die beiden Werte, der siebten und achten Zeile, sind vermutlich deswegen so hoch, da sie bei der höchsten Geschwindigkeit des Schiebers aufgenommen wurden.
Bei dieser hohen Geschwindigkeit beginnt die Zahnradkonstruktion zu wackeln, was in einer zu hohen Zählrate endet.\\
Durch diesen neuen Wert ergeben sich bessere Werte für die Messung des Brechungsindex für Luft
\begin{align*}
  n_{\text{Luft,neu}} &= \input{build/n_luft_neu.tex},\\
  \Delta n_{\text{Luft,neu}} &= \input{build/n_luft_rel_neu.tex}.
\end{align*}
Der ermittelte Wert für $\ce{C4H8}$ ist jedoch sehr weit gefehlt.
Während der Messung ist auffällig gewesen, dass der Druck in dem Gasbehälter permanent dem Normaldruck entgegen strebte.
Es scheint nicht hinreichend gut isoliert zu sein, sodass während der Messung von außen zu viele Luftmolekühle hinein diffundierten, wodurch die Messwerte nach unten hin verfälscht wurden.
