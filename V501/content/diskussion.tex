\section{Diskussion}
\label{sec:Diskussion}

\subsection{Diskussion zum Versuch mit E-Feld}
Das Ergebnis zur Messung der Konstante bezüglich der Ablenkung des Elektronenstrahls im elektrischen Feld lautet
\begin{align*}
  \frac{pL}{2d} &= \input{build/parameter_m6.tex}. \\
\end{align*}
Der exakte Wert für die Konstante der Messapparatur \cite{skript1} ist
\begin{align*}
  \frac{pL}{2d} &= \input{build/parameter_m6_lit.tex}. \\
\end{align*}
Somit ergibt sich erstaunlicherweise eine relativ kleine Abweichung von
\begin{align*}
  \increment{\frac{pL}{2d}} &= \input{build/parameter_m6_rel.tex}, \\
\end{align*}
da eigentlich eine signifikant hohe Abweichung erwartet wird.\\
Die Frequenz der Sinusspannung wird auf
\begin{align*}
  \nu = \input{build/v1.tex},
\end{align*}
mit einer Amplitude von
\begin{align*}
  \hat{U} &= \input{build/U_amp.tex}
\end{align*}
bestimmt.
Die dritte gemessene Frequenz von
\begin{align*}
  \nu_3 = \input{build/v2.tex}
\end{align*}
wird entfernt, da sich dieser Wert dem beobachtbaren Muster nicht anpasst.

\subsection{Diskussion zum Versuch mit B-Feld}
Die empirisch bestimmte spezifische Elektronenladung beträgt
\begin{align*}
  \frac{e_0}{m_0} = \input{build/konstante_mean.tex}.
\end{align*}
Verglichen mit dem Literaturwert \cite{Konstanten} von
\begin{align*}
  \frac{e_0}{m_0} = \input{build/lit.tex},
\end{align*}
ergibt sich eine relative Abweichung von
\begin{align*}
  \increment{\frac{e_0}{m_0}} = \input{build/rel.tex}.
\end{align*}
Ein möglicher Grund für die Abweichung kann in einer fehlerhaften Ausrichtung des Versuchsaufbaus liegen.
Er ist möglicherweise nicht ganz parallel zum Erdmagnetfeld ausgerichtet.\\
Für die Messung der Erdmagnetfeldstärke ergibt sich der Wert
\begin{align*}
  B_{\text{Erde}} = \input{build/erdmagnetfeld_korrigiert.tex}.
\end{align*}
Sie sollte nach Literatur \cite{erde} ungefähr um den Wert
\begin{align*}
  B_{\text{Erde}} = \input{build/erdmagnetfeld_korrigiert_lit.tex}
\end{align*}
liegen.
Daher sollte es ungefähr richtig bestimmt worden sein.
