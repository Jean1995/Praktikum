\section{Versuchsdurchführung}
Vor Beginn der Versuchsreihen wird der Aufbau gemäß Versuchsaufbau verkabelt.
Es muss auf die Erdung der Ablenkplatten geachtet werden.
Nach einer einminütigen Aufwärmzeit der Glühkathode kann die Beschleunigungspannung angelegt und die Kathodenstrahlröhre verwendet werden.
Für jede Messung wird die Spannung $U_c$ der Fokussierungselektroden sowie die Leuchtintensität so gewählt, dass der Elektronenfleck auf dem Schirm gut zu erkennen ist.
\subsection{Untersuchung der Ablenkungseigenschaften des Elektronenstrahls}
Zunächst wird für fünf verschiedene Beschleunigungspannungen zwischen $\SI{250}{\volt}$ und $\SI{400}{\volt}$ die Ablenkung $D$ des Elektronenstrahls bei verschiedenen Ablenkspannungen in Y-Richtung untersucht.
Dabei wird willkürlich der Messnullpunkt auf die oberste Linie des Koordinatensystems gelegt.
Es wird nun gemessen, bei welcher Ablenkspannungen die Koordinantenlinien vom Elektronenstrahl getroffen werden.
\subsection{Nutzung der Kathodenstrahlröhre als einfachen Oszillographen}
Die Kathodenstrahlröhre bildet die Grundlage eines analogen Oszilliographen.
Dazu wird die zu messende Spannung an die Ablenkplatten in Y-Richtung sowie eine Sägezahnspannung an die Ablenkplatten in X-Richtung angelegt.
Es wird eine feste Sinusspannung betrachet und es werden die Frequenzen der Sägezahnspannung notiert, für die eine stehende Welle abgelesen werden kann.
Zudem wird jeweils die Amplitude der Sinusspannung abgelesen.
\label{sec:durchführung}
\subsection{Untersuchung der Ablenkung durch ein Magnetfeld}
Mithilfe der Helmholtzspulen wird der Elektronenstrahl einem Magnetfeld ausgesetzt.
Um die Auswirkung des Erdmagnetfeldes zu eliminieren, wird mithilfe eines Kompasses die Richtung des Erdmagnetfeldes lokalisiert und der Aufbau parallel zu diesem ausgerichtet.
Nun wird für fünf Beschleunigungsspannungen zwischen $\SI{250}{\volt}$ und $\SI{450}{\volt}$ die Ablenkung $D$ in Abhängigkeit von der Magnetfeldstärke $B$ betrachtet.
Dazu wird der Elektronenstrahl zunächst mithilfe der Ablenkplatten auf die oberste Linie des Koordinatennetzes bei ausgeschaltetem Magnetfeld ausgerichtet.
Anschließend wird gemessen, für welchen an der Helmholtzspule angelegten Strom der Elektronenstrahl auf die 9 Koordinatenlinien gerichtet ist.
\subsection{Bestimmung des Erdmagnetfeldes durch dessen Einfluss auf den Elektronenstrahl}
Bei einer Beschleunigungsspannung von $\SI{200}{\volt}$ wird der Elektronenstrahl mit den Ablenkplatten auf den Mittelpunkt des Koordinatennetzes ausgerichtet.
Dabei befindet sich der Versuchsaufbau noch parallel zum Erdmagnetfeld, das Magnetfeld der Helmholtzspule ist ausgeschaltet.
Daraufhin wird der Aufbau um $\SI{90}{\degree}$ gedreht.
Der nun abgelenkte Elektronenstrahl wird durch das Anlegen eines Stromes an die Helmholtzspule wieder auf den Koordinatenmittelpunkt ausgerichtet.
Somit wird das Erdmagnetfeld durch das Magnetfeld der Spulen gerade kompensiert.\\
Da das Erdmagnetfeld unter einem Inklinationswnkel $\phi$ auf den Versuchsaufbau einfällt, muss ebendieser Winkel mit dem Kompass bestimmt werden.
