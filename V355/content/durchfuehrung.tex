\section{Aufbau und Durchführung}
\label{sec:Durchführung}
\subsection{Aufbau}
Die Schwebungen werden mit einem Schwingkreis, wie in Abbildung \ref{fig:1}, untersucht.








%\subsection{Durchführung}
Die Schwebungen werden mit einem Schwingkreis, wie in Abbildung \ref{fig:1} aus der Theorie, untersucht.
Bevor die Messreihen beginnen, müssen gewisse Justierungen getroffen werden.
Um, wie in der Theorie beschrieben, einen möglichst guten Energieaustausch zu gewährleisten, sollten die beiden Kondensatoren die selbe Resonanzfrequenz besitzen.
Dazu wird zuerst die Schaltung aus Abbildung \ref{fig:2} nachgebaut.



Sie besteht aus einem Sinusgenerator und jeweils dazu einen in Reihe geschalteten Kondensator und einer Spule.
Ein Oszilloskop misst nach der Spule an einem Ohmschen Widerstand die Spannung über den Y-Eingang.
Dessen X-Eingang wird zwischen Generator und Kondensator zugesteckt.\\
Eine Resonanz liegt vor wenn die Phase zwischen Generatorspannung und Schwingkreisstrom 0 beziehungsweise $\sumup{\pi}$ ist.
Dies wird über die Lissajous
