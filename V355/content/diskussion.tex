\section{Diskussion}
\label{sec:Diskussion}
Im Folgenden werden die gemessenen Verhältnisse zwischen den Schwebungs- und Resonanzfrequenzen mit den aus den Bauteilen ermittelten verglichen.
Mit einem Ablesefehler von einem Extrema beim gemessenen Wert liegen bis auf 2 Messungen [\ref{tab:1}] alle berechneten Werte im Fehlerintervall.
Die relative Abweichung steigt jedoch bei geringen Kapazitäten von $C_k$ sehr stark an, bis auf $30\%$ beim letzten Wert.
Dies liegt daran, dass die Sinusschwingung bei geringen $C_k$ recht unförmig ist und sehr wenige Extrema aufweist.
Dadurch ist das Ablesen bzw. Abschätzen der Extrema erschwert.\\
Durch das Einrechnen der eigentlichen Kapazität $C_{\text{sp}}$der Spule schmiegen sich die Verhältnisse noch genauer an die abgelesenen Werte an [\ref{tab:2}].
Jedoch zeichnet sich eine größere Abweichung besonders in höheren Werten von $C_k$ ab.
Die relativen Abweichungen sind jedoch im Vergleich zur Rechnung ohne $C_{\text{sp}}$ geringer.\\
Da die Messungenauigkeit $\increment C_k$ verursacht durch $C_k$ sehr klein im Vergleich zum Ablesefehler $\increment N$ der Extrema ist, kann diese vernachlässigt werden.
Die benutzte Methode zur Berechnung des Verhältnisses scheint sich in der Messreihe zu bestätigen.\\
Beim nächsten Aufgabenteil sollten die Fundamentalfrequenzen für verschiedene $C_k$ gemessen werden.
Da sich die Lissajous-Figuren beim ersten Peak $\nu_1$ kaum messbar veränderten, wurde nur ein Wert, $\nu_1 = \SI{30.30}{\kilo\hertz}$, aufgenommen.
Diese kaum messbare Veränderung schließt auf ein Bestätigen der Formel, die besagt, dass $\nu_1$ unabhängig von $C_k$ ist.
Die relative Abweichung von $0,84\%$ ist ziemlich gering und rührt wahrscheinlich von der Einstellschwierigkeit der gewünschten Lissajous-Figur mit Hilfe des Frequenzgenerators.
Die steigenden Werte von $\nu_2$ schließen auf eine Abhängigkeit von $C_k$ und mit zusätzlichem Hinblick auf die berechneten Werte scheint sich die benutzte Formel aus der Auswertung , (\ref{eqn:omega2_neu}), umgerechnet in Frequenzen zu bestätigen.
Jedoch zeigt sich auch hier ein Abweichen der Theoriewerte bei kleinen $C_k$ bis zu $6,24\%$.\\
Im letzten Aufgabenteil werden die Fundamentalfrequenzen für den Kondensator mit kleinster Kapazität anhand des Frequenzspektrums bestimmt.
Die abgelesenen Werte für $\nu_1$ und $\nu_2$ stimmen nach Umrechnung in Frequenzen bis auf einen vernachlässigbar kleinen Fehler mit den im vorherigen Aufgabenteil abgelesenen Frequenzen überein.
Somit unterliegen die so bestimmten Frequenzen den gleichen systematischen Fehler wie bei der vorherigen Methode.
Dementsprechend liefert uns das Frequenzspektrum keine verbesserte Aussage im Hinblick auf die Fundamentalfrequenzen.
Zu den berechneten Strömen $I_{2,1}$ und $I_{2,2}$ haben wir aus Zeitmangel keine Referenzwerte.
