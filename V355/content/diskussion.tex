\section{Diskussion}
\label{sec:Diskussion}
Im Folgenden werden die gemessenen Verhältnisse zwischen den Schwebungs- und Resonanzfrequenzen mit den aus den Bauteilen ermittelten verglichen.
Mit einem Ablesefehler von einem Extrema beim gemessenen Wert liegen bis auf 2 Messungen [\ref{tab:1}] alle berechneten Werte im Fehlerintervall.
Die relative Abweichung steigt jedoch bei geringen Kapazitäten von $C_k$ sehr stark an, bis auf $30\%$ beim letzten Wert.
Dies liegt daran, dass die Sinusschwingung bei geringen $C_k$ recht unförmig ist und sehr wenige Extrema aufweist.
Dadurch ist das Ablesen bzw. Abschätzen der Extrema erschwert.
Durch das Einrechnen der eigentlichen Kapazität $C_{\text{sp}}$der Spule schmiegen sich die Verhältnisse noch genauer an die abgelesenen Werte an [\ref{tab:2}].
Jedoch zeichnet sich eine größere Abweichung besonders in höheren Werten von $C_k$ ab.
Die relativen Abweichung sind jedoch im Vergleich zur Rechnung ohne $C_{\text{sp}}$ geringer.
Da die Messungenauigkeit $\increment C_k$ verursacht durch $C_k$ sehr klein im Vergleich zum Ablesefehler $\increment N$ der Extrema ist, kann diese vernachlässigt werden.
Die benutzte Methode zur Berechnung des Verhältnisses scheint sich in der Messreihe zu bestätigen.
