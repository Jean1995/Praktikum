\section{Diskussion}
\label{sec:Diskussion}
In der Emissionsbestimmung gehen wir davon aus, dass sich der Würfel bei der schwarzen Oberfläche wie ein schwarzer Körper verhält.
Ein schwazer Körper ist jedoch ein idealisiertes Konstrukt, welches es nicht gibt.\\
Zudem war die Thermosäule sehr fehleranfällig.
Sie ist wie im Aufbau beschrieben ein sehr empfindliches Gerät, das sogar besonders stark auf Infrarotstrahlung reagiert.
Demnach ergaben sich große Fehler durch die Wärmestrahlung von Menschengruppen in unmittelbarer Umgebung.
Außerdem wurde am Nebentisch ein Experiment mit offener Herdplatte durchgeführt, die inkonsistent ein- und wieder ausgeschaltet wurde.
Jene zusätzliche Wärmestrahlung wurde von den metallischen Oberflächen (besonders von der glänzenden) des Würfels zur Thermosäule reflektiert.
Wir versuchten besonders bei der Anwesenheit von Menschengruppen zu warten bis sich ein konstanter Wert eingependelt hat.
Währenddessen fiel die Temperatur jedoch wieder relativ schnell, was wiederum zu verzögerten Messungen führte.
Der ständige Temperaturabfall und der damit verbundene Zeitdruck war besonders am Anfang der Messung ein Problem.
Da zu dem Zeitpunkt die Temperatur sehr schnell abstieg, war es äußerst kompliziert, in einem möglichst kleinen Zeitintervall alle vier Seitenflächen zu messen.
Das selbe Problem zeichnete sich auch bei der abstandsabhängigen Messung ab.
Unter derartigem Zeitdruck war es nahezu unmöglich, den Sockel der Thermosäule mit den Handschuhen zu verschieben, da in ihnen die nötige Feinmotorik für diesen Eingriff fehlte.
Daher versuchten wir sie vorsichtig mit den bloßen Händen zu positionieren.
Abgesehen davon ist bei der abstandsabhängigen Messung zu erwähnen, dass mit dem Vergrößern des Abstandes mehr Fehlerquellen in den Fokus der Thermosäule fallen.
