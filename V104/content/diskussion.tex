\section{Diskussion}
\label{sec:Diskussion}
Zunächst soll der nominelle Unterschied der Formeln \ref{eqn:5} und \ref{eqn:8}, also der Relativbewegung vom Empfänger zum Sender bzw. umgekehrt betrachtet werden.
Betrachtet man die Ruhefrequenz
\begin{align*}
  \nu_0 &= \input{build/f_0.tex},
\end{align*}
die der Literatur entnommene Schallgeschwindigkeit bei Raumtemperatur
\begin{align*}
  c = \SI{342.2}{\metre\per\second}
\end{align*}
sowie die maximale Wagengeschwindigkeit $v_{\text{max}}$, so ergibt sich ein sehr kleine Abweichung der beiden Formeln von
\begin{align*}
  \increment{nu} &= \input{build/deltanu.tex}.
\end{align*}
Dementsprechend kann der Unterschied beider Formeln vernachlässigt und für beide Fälle einen linearen Zusammenhang annehmen.\\
Wird der experimentell bestimmte Wert der Schallgeschwindigkeit von
\begin{align*}
  c &= \input{build/c.tex}
\end{align*}
mit dem oben genannten Literaturwert, so ergibt sich eine Abweichung von
\begin{align*}
  \increment{c} &= \input{build/relc.tex}.
\end{align*}
Der Literaturwert befindet sich jedoch nicht im Fehlerintervall.\\
Mithilfe der beiden Plots kann ein linearer Zusammenhang zwischen der Relativgeschwindigkeit und der Frequenzdifferenz bestätigt werden.
Der in Kapitel \ref{sec:aa1} mittels linearer Regression berechnete Wert von
\begin{align*}
  \frac{\nu_0}{c} &= \input{build/propfak_1.tex}
\end{align*}
entspricht dem aus Formel \ref{eqn:5} bestimmten Theoriewert
\begin{align*}
  \frac{\nu_0}{c}_t &= \SI{60.611}{\metre\per\second}.
\end{align*}
Der mittels Schwebungsmethode gemessene Wert von
\begin{align*}
  \frac{\nu_0}{c} &= \input{build/propfak_2.tex}
\end{align*}
weicht zwar stärker von dem Theoriewert ab, jener liegt dennoch weiterhin im Fehlerintervall.
Die hier vorhandene stärkere Abweichung lässt sich damit erklären, dass bei dieser Messung weniger Messwerte aufgenommen werden konnten.
