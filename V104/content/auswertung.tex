\section{Auswertung}
\label{sec:Auswertung}







\begin{table}[H]
  \centering
  \caption{Wagengeschwindigkeiten}
  \label{tab:Geschwindigkeiten}
  \sisetup{table-format=1.2}
  \begin{tabular}{c c c}
    \toprule
    {$\text{Gang}$} & {$v [\si{\centi\metre\per\second}]$} & {$\increment v [\si{\centi\metre\per\second}]$}\\
    \midrule
    \input{build/geschwtabelle.tex}
    \bottomrule
  \end{tabular}
\end{table}



\begin{align*}
  \lambda_0 &= \input{build/wl.tex}
\end{align*}



\begin{align*}
  c &= \input{build/c.tex}
\end{align*}

\begin{align*}
  \frac{1}{\lambda_0} &= \input{build/einsdurchwl.tex}
\end{align*}








%\begin{figure}
%  \centering
%  \includegraphics{plot.pdf}
%  \caption{Plot.}
%  \label{fig:plot}
%\end{figure}
%
%\begin{table}
%  \centering
%  \caption{Beispieltabelle}
%  \label{tab:tabelle_beispiel}
%  \sisetup{table-format=1.2}
%  \begin{tabular}{c c}
%    \toprule
%    {$a [\si{\second}]$} & {$b [\si{\kelvin}]$}\\
%    \midrule
%    1.0000  & 11.00 \\
2.0000  & 12.00 \\
3.0000  & 13.00 \\
4.0000  & 14.00 \\
5.0000  & 15.00 \\
6.0000  & 16.00 \\
7.0000  & 17.00 \\
8.0000  & 18.00 \\
9.0000  & 19.00 \\
10.0000 & 20.00 \\

%    \bottomrule
%  \end{tabular}
%\end{table}
%
%Es ergibt sich
%\begin{align}
%  a &= (0 \pm 0) ~ \si{\joule\per\kelvin\per\gram}
 \\
%\end{align}
