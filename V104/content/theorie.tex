\section{Theorie}
\label{sec:Theorie}
Der Doppler-Effekt wird beobachtet, wenn sich der Sender einer Welle und dessen Empfänger relativ zueinander bewegen.
Der Empfänger jener Welle nimmt diese sobald eine Relativgeschwindigkeit $v$ vorliegt mit veränderter Frequenz war.
Die Frequenzänderung $\increment \nu$ muss jedoch in Bezug auf zwei Fälle untersucht werden, wenn sich die Welle in einem Medium ausbreitet.\\
\subsection{Empfänger bewegt sich relativ zum Ausbreitungsmedium}
%Im ersten Fall bewegt sich der Empfänger relativ zum Ausbreitungsmedium.
Die in diesem Fall ruhende Quelle sende die Frequenz $\nu_0$ aus, welche ohne Relativbewegung des Empfängers auch als solche wahrgenommen wird.
Bewegt er sich aber, dann nimmt er je nach Bewegungsrichtung mehr oder weniger Wellenzüge $\increment z$ in der Zeit $\increment t$ wahr.
Es folgt der Zusammenhang
\begin{equation}
  \increment z = \frac{\increment t v}{\lambda_0} \label{eqn:1},
\end{equation}
wobei $\lambda_0$ der Wellenlänge der ausgesendeten Welle im System des Senders entspricht.
An einem ruhenden Empfänger laufen jedoch nur
\begin{equation}
  \increment n = \increment t \nu_0 \label{eqn:2}
\end{equation}
Schwingungen vorbei.
Insgesamt nimmt er also
\begin{equation}
  \increment n + \increment z = \increment t \left( \nu_0 + \frac{v}{\lambda_0}\right) \label{eqn:3}
\end{equation}
Schwingungen wahr.
Die empfangene Frequenz lautet demnach
\begin{equation}
  \nu_{\text{E}} = \nu_0 + \frac{v}{\lambda_0} \label{eqn:4}
\end{equation}
und in Bezug auf die Ausbreitungsgeschwindigkeit
\begin{equation}
  \label{eqn:4.5}
  c = \nu_0 \lambda_0
\end{equation}
folgt
\begin{equation}
  \nu_{\text{E}} = \nu_0 \left(1 + \frac{v}{c} \right). \label{eqn:5}
\end{equation}
Die gesuchte Frequenzänderung $\increment \nu$ beträgt folglich
\begin{equation}
  \increment \nu = \nu_0 \frac{v}{c}. \label{eqn:5}
\end{equation}
Für eine Bewegung zum Sender $(v>0)$ ist die empfangene Frequenz $\nu_{\text{E}}$ also größer als die Senderfrequenz $\nu_0$, entsprechend anders herum kleiner.
\subsection{Sender bewegt sich relativ zum Ausbreitungsmedium}
Da der Sender nun in einer Zeit $\increment t$ eine gewisse Strecke zurücklegt, erscheint die Wellenlänge, die beim ruhenden Beobachter ankommt, um
\begin{equation}
  \increment \lambda = \frac{v}{\nu_0} \label{eqn:6}
\end{equation}
verkürzt.
Für die wahrgenommene Frequenz folgt entsprechend der Zusammenhang
\begin{equation}
  \nu_{\text{Q}} = \frac{c}{\lambda_0-\increment \lambda}, \label{eqn:7}
\end{equation}
bzw.
\begin{equation}
  \nu_{\text{Q}} = \nu_0 \frac{1}{1-v/c}. \label{eqn:8}
\end{equation}
Wird dieser Ausdruck entwickelt, zeigt sich, dass je nach Art der Bewegung, unterschiedliche Frequenzen beim Beobachter ankommen.
Ist der Betrag der Relativgeschwindigkeit sehr viel kleiner als die Ausbreitungsgeschwindigkeit wird der Unterschied zwischen (\ref{eqn:5}) und (\ref{eqn:8}) beliebig klein.
\subsection{Der Doppler-Effekt bei elektromagnetischen Wellen}
Wenn sich die Ausbreitungsgeschwindigkeit nun der Lichtgeschwindigkeit annähert, muss ein relativistischer Ansatz gewählt werden.
Dies führt auf eine wahrgenommene Frequenz von
\begin{equation}
  \nu' = \nu_0 \frac{\sqrt{1-v²/c_0²}}{1-v/c_0}. \label{eqn:9}
\end{equation}
Es besteht nun kein Unterschied mehr, ob sich nun der Sender oder der Empfänger relativ zum Ausbreitungsmedium bewegt.
Durch diesen Zusammenhang kann z.B auf eine Relativbewegung zwischen der Erde und Fixsternen oder Galaxien geschlossen werden.
