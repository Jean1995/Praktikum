\section{Theorie}
\label{sec:Theorie}

Das Franck-Hertz-Versuch geht es darum, die Strukturauflösung der Elektronenhülle von Atomen zu untersuchen.
Dies erfolgt über ein Elektronenstoßexperiment.\\
Aus einer Quelle beschleunigte Elektronen stoßen dabei auf Hg-Atome, wobei zwischen zwei Fällen unterschieden wird.
Zunächst können elastische und unelastische Stöße zwischen den ausgesandten Elektronen und dem Hg-Atom auftreten.\\
Der erste Fall führt auf Grund des hohen Massenunterschiedes zu einem hinreichend kleinen Energieverlust, jedoch einer großen Richtungsänderung des Elektrons.
In letzterem Fall wird eine diskrete Energiemenge auf ein Hüllenelektron des Hg-Atoms übertragen.
Durch diese Energiemenge steigt das Hg-Atom vom ursprünglichen Energiezustand $E_0$ in einen angeregten Zustand der Engerie $E_1$ auf.
Die Information über die Differenz der Energiezustände spiegelt sich in der Differenz der Energie des Stoßpartners in der Gleichung

\begin{equation}
\frac{m_0 \cdot v²_{\text{vor}}}{2} - \frac{m_0 \cdot v²_{\text{nach}}}{2} = E_1 - E_0 \label{eqn:1}
\end{equation}

wieder.
Um diese Information wahrzunehmen, muss demnach die kinetische Energie des Elektrons vor und nach dem Stoß bekannt sein.
Befindet sich nun ein Hg-Atom in einem angeregten Zustand emittiert es nach einer Relaxationszeit der Größenordnung $\SI{e8}{\second}$ ein Lichtquant der Energie

\begin{equation}
  h\nu = E_0 - E_1. \label{eqn:2}
\end{equation}

Dabei beschreibt $h$ das Plancksche-Wirkungsquantum und $\nu$ die Frequenz eben dieses Lichtquants.

\subsection{Systematischer Aufbau und die Gegenfeldmethode}

\subsection{Störeffekte}



















\cite{sample}

% 2x2 Plot
% \begin{figure*}
%     \centering
%     \begin{subfigure}[b]{0.475\textwidth}
%         \centering
%         \includegraphics[width=\textwidth]{Abbildungen/Schaltung1.pdf}
%         \caption[]%
%         {{\small Schaltung 1.}}
%         \label{fig:Schaltung1}
%     \end{subfigure}
%     \hfill
%     \begin{subfigure}[b]{0.475\textwidth}
%         \centering
%         \includegraphics[width=\textwidth]{Abbildungen/Schaltung2.pdf}
%         \caption[]%
%         {{\small Schaltung 2.}}
%         \label{fig:Schaltung2}
%     \end{subfigure}
%     \vskip\baselineskip
%     \begin{subfigure}[b]{0.475\textwidth}
%         \centering
%         \includegraphics[width=\textwidth]{Abbildungen/Schaltung4.pdf}    % Zahlen vertauscht ... -.-
%         \caption[]%
%         {{\small Schaltung 3.}}
%         \label{fig:Schaltung3}
%     \end{subfigure}
%     \quad
%     \begin{subfigure}[b]{0.475\textwidth}
%         \centering
%         \includegraphics[width=\textwidth]{Abbildungen/Schaltung3.pdf}
%         \caption[]%
%         {{\small Schaltung 4.}}
%         \label{fig:Schaltung4}
%     \end{subfigure}
%     \caption[]
%     {Ersatzschaltbilder der verschiedenen Teilaufgaben.}
%     \label{fig:Schaltungen}
% \end{figure*}
