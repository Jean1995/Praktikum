\subsection{Durchführung}
\label{sec:durchführung}
Zunächst wird das Innere der Röhre auf die gewünschte Temperatur gebracht, indem der Temperaturregler hochgestellt wird.
Dies resultiert in einem ansteigenden Heizstroms.
Sobald die gewünschte Temperatur erreicht ist, wird der Temperaturregler heruntergeregelt, so dass der Heizstrom wieder abnimmt und die Temperatur konstant gehalten wird.
Der XY-Schreiber wird vor jedem Versuchsabschnitt justiert, so dass die gewünschten Messbereiche graphisch dargestellt werden können.

\subsubsection{Bestimmung der integralen Energieverteilung}
Um die Energieverteilung der Elektronen zu bestimmen wird die Beschleunigungsspannung auf einen konstanten Wert von $U_B = \SI{11}{\volt}$ eingestellt, die Bremsspannung wird von $\SI{0}{\volt}$ bis zu einem Verschwinden des Auffängerstroms durchlaufen.
Der Strom wird hierbei auf die Y-Achse, die Bremsspannung auf die X-Achse des XY-Schreibers abgetragen.
Es wird eine Messung bei $T = \SI{26.1}{\celsius}$ sowie eine weitere Messung bei $T = \SI{145.5}{\celcius}$ durchgeführt.

\subsubsection{Aufnahme der Frank-Hertz-Kurven}
Bei einer konstanten Bremsspannung von $U_A = \SI{1}{\volt}$ wird die Beschleunigungsspannung von $U_B = \SI{0}{\volt}$ bis ca. $U_B = \SI{55}{\volt}$ durchlaufen.
Der Auffängerstrom wird widerum auf die Y-Achse gegeben, die Beschleunigungsspannung auf die X-Achse.
Es werden Kurven bei $T = \SI{161}{\celsius}$ sowie $T = \SI{178}{\celsius}$ aufgenommen.

\subsubsection{Bestimmung der Ionisierungsspannung}
Um die Ionisierungsspannung zu ermitteln, wird eine Bremsspannung von $U_A = \SI{-30}{\volt}$ angelegt.
Zudem wird die Beschleunigungsspannung erhöht, bis ein starker Ausschlag bei dem Auffängerstrom zu beobachten ist.
Dementprechend wird $U_B$ auf die X-Achse sowie der Auffängerstrom auf die Y-Achse abgetragen.
Der Versuch wird bei $T = \SI{106.6}{\celsius}$ durchgeführt.
