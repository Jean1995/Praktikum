\section{Diskussion}
\label{sec:Diskussion}

Die Ergebnisse zur Untersuchung der Grenzfrequenzen mittels Durchlasskurve und Theorie lauten
\begin{align*}
  \nu_{1} &= \input{g1.tex}, \\
  \nu_{1, \text{theorie}} &= \input{g1_t.tex}, \\
  \nu_{2} &= \input{g2.tex}, \\
  \nu_{2, \text{theorie}} &= \input{g2_t.tex}.
\end{align*}
Es zeigt sich eine relative Abweichung zum Theoriewert von
\begin{align*}
  \increment\nu_{1} &= 1,3\:\%, \\
  \increment\nu_{2} &= 0,9\:\%.
\end{align*}
Die Schwankung der aufgenommenen Durchlasskurve, dargestellt in Abbildung \ref{fig:durchlass}, lassen sich dadurch erklären, dass der Wellenwiderstand frequenzabhängig ist.
Dies wird jedoch bei dieser Messung vernachlässigt.
Es entstehen daher Teilreflexionen, welche sich in jenen Schwankungen äußern.
Dennoch lassen sich die Grenzfrequenzen gut erkennen.\\
Die Abbildung \ref{fig:dispersion_fertig} spiegelt die beiden Äste der Dispersionsrelation wieder.
Trotz Schwierigkeiten beim Justieren der gewünschten Lissajous-Figuren, was sich letztendlich im Fehlen eines Wertes bemerkbar macht, liegen die gemessenen Werte auf den berechneten Theorieästen.
Bei hohen Frequenzen zeigt sich erneut die Schwierigkeit des genauen Ablesens der Werte.\\
Ein systematischer Fehler liegt im benutzten Frequenzgenerator, welcher die Frequenz konstant abfallen lässt.
%Aus welchem Grund auch immer ._________.
Es ist generell problematisch mit diesem Gerät exakt eine gewünschte Frequenz einzustellen.
Dies könnte ein Grund für die konstante Abweichung der aus den Messdaten berechneten Phasengeschwindigkeiten zu den Theoriekurven sein.
Auffallend ist



%Zur a:
%Die Kurve sieht so komisch aus, weil der Wellenwiderstand frequenzabhängig ist aber ja nicht geändert wird. Also haben wir Reflexionen.
%Die Werte sehen aber echt geil aus.
%
%Zur b:
%Höhere Frequenzen waren Kacke abzulesen. Deswegen sind die Werte nicht so genau.
%Sehen aber trotzdem geil aus.
%
%Zur c:
%
%
%Zur d:
%Spannung nimmt nach hinten ab: Wegen ohmschen Verlusten? (Macht das Sinn, weil wegen Wechselspannung?)
%
%Nicht 0 am Spannungsbauch weil wegen Reflexion weil wegen frequenzabhängigen Wellenwiderstand
